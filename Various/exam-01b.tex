%%%%%%%%%%%%%%%%%%%%%%%%%%%%%%%%%%%%%%%%%%%%%%%%%%%%%%%%%%%%%%%%%%%%%%%%%%%%
% FILE         : exam-01b.tex
% AUTHOR       : Peter C. Chapin
% LAST REVISED : 2012-04-11
% SUBJECT      : Exam #1 for CIS-3152
%
% Send comments or bug reports to:
%
%       Peter C. Chapin
%       Computer Information Systems Department
%       Vermont Technical College
%       Randolph Center, VT 05061
%       PChapin@vtc.vsc.edu
%%%%%%%%%%%%%%%%%%%%%%%%%%%%%%%%%%%%%%%%%%%%%%%%%%%%%%%%%%%%%%%%%%%%%%%%%%%%

\documentclass[12pt]{examdesign}

% Options.
\Fullpages
\NoRearrange
% \NoKey
\NumberOfVersions{1}

% Class and exam information.
\class{CIS-3152}
\examname{Exam \#1}

% This setting seems to be ignored (in the answer key at least).
% \setlength{\parskip}{1.75ex plus0.5ex minus0.5ex}

\begin{document}

% I can't seem to be able to use namedata@vspace below (see documen- tation). I can use the
% default \namedata but I can't seem to redefine \namedata to suite my needs.
%
\begin{examtop}
  \parbox{3in}{\classdata \\
               \examtype, Form: \fbox{\textsf{\Alph{version}}}}
  \hfill
  \parbox{3in}{\normalsize Name: \hrulefill \\[2.0ex]
                           Date: \hrulefill }
  \bigskip
\end{examtop}

\begin{shortanswer}

  % Sockets programming.
  \begin{question}
    The call to \texttt{connect()} in a typical sockets based TCP program uses a second argument
    that looks like \texttt{(struct sockaddr *) \&server\_address} where
    \texttt{server\_address} is a \texttt{struct sockaddr\_in}. Why is the complicated type
    conversion necessary? In other words: why wasn't \texttt{connect()} declared to take a
    \texttt{struct sockaddr\_in *} in the first place?

    \begin{answer}
      The sockets API was designed to work with many different protocol families. Each protocol
      family has its own way of addressing nodes and thus has its own socket address structure.
      The \texttt{connect()} function must be able to take a pointer to any of these address
      structure types. To deal with that in C, \texttt{connect()} has been declared as taking a
      pointer to a generic address structure with the understanding that the caller can send in
      a pointer to any of the supported structures (by using an appropriate type cast).
    \end{answer}
  \end{question}

%\pagebreak

  % TCP state diagram.
  \begin{question}
  	Answer two of the three questions below.
    \begin{enumerate}
    \item When a client program calls \texttt{connect()}, what state or states in the TCP state
      diagram does the connection touch?\footnote{By ``touch'' I mean ``pass through'' or ``end
        up in.''} You can assume normal, error free behavior.
    \item What is the difference between ``active close'' and ``passive close?''
    \item What sequence of states does the TCP connection touch when the side doing the passive
      close calls \texttt{close()}? You can assume normal, error free behavior.
    \end{enumerate}
    
    \begin{answer}
    \begin{enumerate}
    
    \item When calling \texttt{connect()} the client passes through the SYN\_SENT state and ends
      up in the ESTABLISHED state.
    \item The side that does the active close is the side that initially calls \texttt{close()}
      to end a connection. This side sends the first FIN segment. The side that does the passive
      close acknowledges the initial FIN segment. It then later calls \texttt{close()} as well
      to send a FIN segment of its own.
    \item When the passive side calls \texttt{close()} it starts in CLOSE\_WAIT and passes
    through LAST\_ACK.
    \end{enumerate}
    \end{answer}
  \end{question}

%\pagebreak

  % TCP internals.
  \begin{question}
    Answer two of the three questions below.
    \begin{enumerate}
    \item TCP is said to be a ``windowed protocol.'' What does that mean and why is that
      important?
    \item What is the difference between the receiver window and the congestion window?
    \item To decide if it should retransmit a segment, TCP uses ``intelligent timeouts.'' What
      is intelligent about them?
    \end{enumerate}

    \begin{answer}
    \begin{enumerate}
    \item Instead of waiting for each segment to be acknowledged separately, TCP transmits as
      many segments as it can to fill a ``window'' of data. When it receives an acknowledgment
      for the first segment sent, the window moves down the data stream exposing some fresh data
      at the other end. This is important because it allows a lot of data to be in transit on
      the network and thus allows two nodes to communicate efficiently despite high network
      latencies.
    \item The receiver window reflects the amount of available space in the receiver's buffer.
      The congestion window is the sender's estimate of the network's capacity.
    \item Intelligent timeouts adapt to the specifics of a particular connection and can even
      change over the lifetime of the connection. TCP estimates the mean and standard deviation
      of the round trip time and uses that estimate as the basis for setting the current time
      out interval.
    \end{enumerate}
    \end{answer}
  \end{question}

%\pagebreak

  % UDP
  \begin{question}
    UDP is an unreliable protocol. However, some services are most appropriately implemented as
    UDP services rather than using the more reliable TCP (DNS and TFTP are like this). Give at
    least one \emph{advantage} that UDP has over TCP in these cases. Under what conditions is
    the unreliability of UDP not an issue?

    \begin{answer}
      UDP's advantage over TCP is its lower overhead. This is particularly significant if the
      messages being exchanged are small. UDP works well when a client's request and the
      server's reply both fit into a single datagram, and when the service being provided has
      the property of being idempotent. UDP is also a good choice in cases where its
      unreliability is not an issue such as streaming media (where lost data degrades the
      quality of the service but does not cause an outright failure of the service).
    \end{answer}
  \end{question}

%\pagebreak

  \begin{question}
    A program that uses UDP will use the \texttt{sendto()} and \texttt{recvfrom()} functions
    instead of the \texttt{send()}/\texttt{recv()} or \texttt{write()}/\texttt{read()} functions
    that a TCP based program would use. The \texttt{sendto()} and \texttt{recvfrom()} functions
    both take a pointer to a sockaddr structure as a parameter. What is the purpose of this
    parameter and why do the UDP functions require it while the TCP functions do not?

    \begin{answer}
      The address of the remote host is provided to (or from) TCP when the connection is
      established (during \texttt{connect()} for the client or \texttt{accept()} for the
      server). Since that address remains the same for the duration of the connection there is
      no need to provide it again during the read/write operations.

      In contrast UDP does not have connections. Thus the address of the remote host must be
      provided to (or obtained from ) each read/write operation separately. This is the purpose
      of the sockaddr structure mentioned above.
    \end{answer}
  \end{question}

%\pagebreak

  % IPv6
  \begin{question}
    BONUS (5pts) There are three general reasons why IPv6 was developed. What are they? For full
    credit you should give a sentence or two of explanation for each reason; avoid single word
    answers.\footnote{However, your explanations can be short; you don't need to fill the page!}

    \begin{answer}
    \begin{enumerate}
    \item The world is running out of IPv4 addresses (for example, due to inefficient allocation
      together with the large number of new network users).
    \item The backbone routers were getting overloaded because of the large routing tables they
      had to search for each packet. IPv6 has features to simplify routing.
    \item IPv4 is quite old and lacks features now considered important: such as support for
      security, mobile devices, and quality of service controls to name a few.
    \end{enumerate}
    \end{answer}
  \end{question}

\end{shortanswer}
\end{document}
