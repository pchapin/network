%%%%%%%%%%%%%%%%%%%%%%%%%%%%%%%%%%%%%%%%%%%%%%%%%%%%%%%%%%%%%%%%%%%%%%%%%%%
% FILE    : lab-template.tex
% SUBJECT : Lab #3 report. This is intended to serve as a sample.
% AUTHOR  : (C) Copyright 2010 by Peter C. Chapin <PChapin@vtc.vsc.edu>
%
%%%%%%%%%%%%%%%%%%%%%%%%%%%%%%%%%%%%%%%%%%%%%%%%%%%%%%%%%%%%%%%%%%%%%%%%%%%

% ++++++++++++++++++++++++++++++++
% Preamble and global declarations
% ++++++++++++++++++++++++++++++++
\documentclass{article}

% --------
% Packages
% --------
\usepackage[pdftex]{graphicx}
\usepackage{listings}
\usepackage{hyperref}
\usepackage{url}

% The following are settings for the listings package
\lstset{language=C,
        basicstyle=\small,
        stringstyle=\ttfamily,
        commentstyle=\ttfamily,
        xleftmargin=0.25in,
        showstringspaces=false}


% ------------------
% Layout adjustments
% ------------------
% Avoid changing the layout; it is usually set via the document class (or in packages).
%\pagestyle{headings}
%\setlength{\parindent}{0em}
%\setlength{\parskip}{1.75ex plus0.5ex minus0.5ex}


% ------
% Macros
% ------
\newcommand{\filename}[1]{\texttt{#1}}
\newcommand{\command}[1]{\texttt{#1}}


% +++++++++++++++++++
% The document itself
% +++++++++++++++++++
\begin{document}

% ----------------------
% Title page information
% ----------------------
\title{CIS--3152 Lab \#3\\Transmission Control Protocol}
\author{Peter C. Chapin\thanks{PChapin@vtc.vsc.edu}\\
  Vermont Technical College}
\date{February 3, 2010}
\maketitle

% --------
% Abstract
% --------
\begin{abstract}
  Using Wireshark I observed the TCP traffic between my quote of the day server and my quote of
  the day client. I recorded and interpreted the TCP segment headers and investigated some of
  the states in the TCP state diagram. With the exception of some details in the behavior of the
  client's window size, all observations agreed with my expectations.
\end{abstract}

% ------------
% Main Content
% ------------

\section{Introduction}
\label{sec:introduction}

In this lab I used the Windows quote of the day client and the Linux quote of the day server
that I wrote for Lab \#2. I observed the TCP traffic between the two programs using Wireshark on
a Windows workstation. Because I was using a network analyzer on the Windows station, it was
essential to run one of the two programs (the client in my case) on the same station. Otherwise
the network analyzer would not have been able to see the communication.

See the lab handout for information about the specific steps taken in this lab.

\section{Results}
\label{sec:results}

I observed the TCP traffic between the client and server programs using the network analyzer.
What follows is a detailed listing of that traffic. The client was running on a WinXP laptop.
The server was running on a Linux system.

\begin{verbatim}
Segment #1: Client to server...
        Source port = 1230
        Destination port = 9000
        Sequence number = 514042259
        Ack number = 0
        Flags: SYN
        Window = 16384
        Options: MSS = 1460 (typical for ethernet)
        Data: 0 bytes
\end{verbatim}

This first segment going from client to server indicates that the client is doing the active
open. This segment was sent in response to the client's call to connect().

\begin{verbatim}
Segment #2: Server to client...
        Source port = 9000
        Destination port = 1230
        Sequence number = 2851385579
        Ack number = 514042260
        Flags: ACK, SYN
        Window = 5840
        Options: MSS = 1460
        Data: 0 bytes
\end{verbatim}

Notice that the server acknowledges the client's sequence number plus one. The SYN segment
consumes one byte's worth of sequence number space. It is also interesting to notice that the
sequence numbers and window sizes in the two different directions are unrelated. For brevity I
leave out the port addresses in all the following segments.

\begin{verbatim}
Segment #3 Client to server... completion of connection handshake.
        Sequence number = 514042260
        Ack number = 2851385580
        Flags: ACK
        Window = 17520
        Data: 0 bytes
\end{verbatim}

After sending this segment the client's call to connect() should have returned. After receiving
this segment the server's call to accept() should have returned. Notice that this segment also
ACKs just past the server's initial sequence number. Notice also that the client has adjusted
its window size upwards. I do not know the reason for this.

\begin{verbatim}
Segment #4 Server to client... the quote!
        Sequence number = 2851385580
        Ack number = 514042260
        Flags: ACK, PSH
        Window = 5840
        Data: 57 bytes
\end{verbatim}

This segment contains the complete quote. It was sent in response to the server's write() call.
The PSH flag indicates to the client TCP that it should deliver this data to the client
application (if it has been buffering data). That makes sense since there will not be any more
data forthcoming.

\begin{verbatim}
Segment #5 Server to client... closing the connection.
        Sequence number = 2851385637
        Ack number = 514042260
        Flags: ACK, FIN
        Window = 5840
        Data: 0 bytes
\end{verbatim}

The server does not wait for the previous segment to be acknowledged. The server application has
called close() and the server sends a FIN segment at once. Since the server is the first to send
a FIN segment, the server is the one doing the active close. Notice that the sequence number of
this segment is 57 bytes past that of the last one---a reflection of the 57 bytes contained in
the earlier segment.

\begin{verbatim}
Segment #6 Client to server... ACK
        Sequence number = 514042260
        Ack number = 2851385638
        Flags: ACK
        Window = 17463
        Data: 0 bytes
\end{verbatim}

With this segment the client acknowledges both the server's data segment and the server's FIN
segment. Notice that the FIN also takes up one byte's worth of sequence number space. The client
is showing a window that is 57 bytes less than it was. This suggests that the 57 data bytes are
still in the client's TCP buffer and have not yet been read by the client application.

\begin{verbatim}
Segment #7 Client to server... closing the connection
        Sequence number = 514042260
        Ack number = 2851385638
        Flags: ACK, FIN
        Window = 17463
        Data: 0 bytes
\end{verbatim}

The client application has processed the data and has called closesocket(). The client's TCP
sends its own FIN segment. Notice that the window size is still the reduced to 17463 bytes. I
would have expected the client's TCP to have restored the window to its prior, larger size after
the client application finished processing the data.

\begin{verbatim}
Segment #8 Server to client... finalizing the close
        Sequence number = 2851385638
        Ack number = 514042261
        Flags: ACK
        Window = 5840
        Data: 0 bytes
\end{verbatim}

With this segment the connection fully ends.

I also used netstat on the server machine to observe some of TCP states. For example, after the
server blocked in accept() but before a client connected the server was shown in the LISTEN
state. I had to use the -a option with netstat to see this. I included some code in the server
that caused it to pause after a connection was made but before it sent any data. This allowed me
to see the server in the ESTABLISHED state (the client was also in the ESTABLISHED state at this
time). I included some code in the client that caused it to pause after it had received all of
the data but before it closed the connection. This allowed me to see the server in the
FIN\_WAIT\_2 state. In that state the server had already closed but was waiting for the client's
corresponding FIN segment. At the same time the client was in the CLOSE\_WAIT state.

Most of the states in the TCP state diagram could not be easily observed. This was because they
are entered and left entirely while my programs were in system calls. Since I was only able to
pause my programs when control was in my code, that prevented me from viewing certain TCP
states.

\section{Conclusion}
\label{sec:conclusion}

The TCP traffic between the client and server behaved as I expected. All sequence numbers,
segment sizes, and flags were consistent with documented TCP standards. The only unexplained
detail I observed was with the client's window size.

The TCP states I observed on the client and server were also in agreement with my expectation.
Although most states were unobservable in this experiment, those that were occured at the
appropriate times.

\end{document}
