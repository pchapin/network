%%%%%%%%%%%%%%%%%%%%%%%%%%%%%%%%%%%%%%%%%%%%%%%%%%%%%%%%%%%%%%%%%%%%%%%%%%%
% FILE    : lab01.tex
% AUTHOR  : (C) Copyright 2010 by Peter C. Chapin <Peter.Chapin@vtc.vsc.edu>
% SUBJECT : First lab for CIS-3152. Introduction to the tools.
%
% In this lab the students will experiment a little with some of the tools they will need to use
% in this course. This is a lab that can be done with zero lecture support and yet is reasonably
% worthwhile.
%%%%%%%%%%%%%%%%%%%%%%%%%%%%%%%%%%%%%%%%%%%%%%%%%%%%%%%%%%%%%%%%%%%%%%%%%%%

% ++++++++++++++++++++++++++++++++
% Preamble and global declarations
% ++++++++++++++++++++++++++++++++
\documentclass[twocolumn]{article}

%\pagestyle{headings}
%\setlength{\parindent}{0em}
%\setlength{\parskip}{1.75ex plus0.5ex minus0.5ex}

% +++++++++++++++++++
% The document itself
% +++++++++++++++++++
\begin{document}

% ----------------------
% Title page information
% ----------------------
\title{CIS--3152 Lab \#1\\Lab Tools}
\author{Peter C. Chapin\thanks{Peter.Chapin@vtc.vsc.edu}\\
  Vermont Technical College}
\date{Last Revised: January 2, 2010}
\maketitle

\section*{Introduction}

In the lab section of this course you will be using a number of software tools. The purpose of
this first lab is to introduce you to these tools so that when you later have to use them in a
serious way you will have some familiarity with them. In particular, in this lab you will
experiment with

\begin{enumerate}

\item \textit{Visual Studio 2008}

\item \textit{X-Win32}

\item \textit{gcc}

\item \textit{Wireshark}

\item \textit{Subversion}

\item \textit{\LaTeX}

\end{enumerate}

In addition you may wish to optionally experiment with

\begin{enumerate}

\item \textit{Open Watcom}

\item \textit{Eclipse}

\end{enumerate}

This lab is intended to be fun. Feel free to experiment with any of the tools as much as you
like.

Note that it is possible for you to use all of the tools above on your own machine. Wireshark,
Subversion, Open Watcom, and Eclipse are open source programs. The department's MSDN
subscription allows you, as a student in our department, to use Visual Studio Professional on
your own machine\footnote{You can also download the free Express version of Visual Studio
  directly from Microsoft.}. X-Win32 is licensed by network address; you can install it on your
own machine but it won't function outside the 155.42.*.* network. If you are interested in
installing any or all of the tools above on your own computer, let me know and I will be happy
to assist you.

\section{Visual Studio 2008}

You will be doing a considerable amount of C and C++ programming in this course. Some of this
programming will be done using gcc on the Linux system \texttt{lemuria} that you will access
remotely. Some of the programming will be done using Visual Studio on the Windows machines in
the lab itself.

\subsection*{Procedure}

\begin{enumerate}

\item On the class web site you will find a document describing how to set up a project using
  Visual Studio 2005. Read over that document and set up a project named \texttt{daytime}
  containing the daytime client program (on the class web site). Note that even though you are
  using Visual Studio 2008 in lab, the procedure for creating a project is essentially identical
  with that for Visual Studio 2005.

\item Add a new project to your daytime solution that contains the daytime server program (on
  the class web site). When you ``build the solution'' both the client and the server should be
  compiled.

\item Run the server on one of your allocated ports (see the class web site for a list of the
  allocations). Run the client in a separate window and verify that it works as expected.

\item Use the Visual Studio debugger to single step the client, inspect the value of variables,
  etc.

\item Visual Studio comes with a document reader that gives you access to the
  MSDN\footnote{Microsoft Developer's Network} library. This library provides extensive
  information on all aspects of Windows programming. Among other things, it serves as the ``man
  pages'' for Windows system functions. As an example use the document reader to search for
  information on the \texttt{socket()} function in the Platform SDK. Also look up
  \texttt{WSASocket()} and compare the two functions. In Windows the \texttt{WSASocket()}
  function is primitive and the \texttt{socket()} function is built on top of it.

\end{enumerate}

\section{X-Win32}

Quite a bit of the programming you will do in this course will be on the department's Linux host
\texttt{lemuria}. Note that \texttt{lemuria} is a multi-core Intel 64 system.

Although it is perfectly acceptable to interact with \texttt{lemuria} using a text mode terminal
program such as \texttt{PuTTY}, you may find it more interesting to use an X server to get a
full graphical desktop on the system. In this part of the lab you will experiment with that.

\subsection*{Procedure}

On the class web site you will find a document that describes how to use the X-Win32 server to
access the department's Linux hosts. Review that document and then log into \texttt{lemuria}
using an X session. Be sure to log out of the host before closing X-Win32. Note that you may
have to install the license key into your session. This is because X-Win32 stores licenses on a
per-user basis. Copy and paste the key on the web page into the X-Win32 license tool.

\section{gcc}

You should also be comfortable with programming in the Linux environment. To practice that, do
the following.

\begin{enumerate}

\item Log into \texttt{lemuria} (using either X or PuTTY) and download the Linux versions of the
  daytime client and server programs, along with their associated make file. You can use this
  make file as a skeleton for your own programs.

\item Using the \texttt{make} utility, compile the two programs. Test them by running the server
  on one of your allocated ports and then connecting to it with the client using the loopback
  address of 127.0.0.1.

\item Use your Windows client program to connect to your Linux server.

\item Use the \texttt{gdb} debugger to single step and observe variables in your server while it
  is handling a client.

\item Use the Linux manual pages to look up information on the \texttt{socket()} function.

\end{enumerate}

\section{Wireshark}

One of the tools that you will be using regularly is a network analyzer\footnote{Also called a
  ``protocol analyzer.''}. This tool allows you to, among other things, observe the traffic on
the network and to decode every frame. You will use the network analyzer to observe normal
network behavior and to gain insight into how various network protocols actually work. You will
also find it a valuable tool for debugging your network programs. The network analyzer will play
the same role for you in this course as an oscilloscope plays in an electronics course. In this
course you will be using the network analyzer called \textit{Wireshark}.

\subsection*{Procedure}

You will be particularly interested in the analyzer's ability to decode and filter frames. Note
that the analyzer distinguishes between capture filters and display filters. A capture filter
limits the frames that are stored in the analyzer's buffer. A display filter limits the frames
that are displayed. Capture filters reduce the amount of memory needed to hold all the frames in
a given session, but capturing everything and using display filters allows you to easily change
your ``view'' of the session after the session has ended.

Be aware that the lab we are using is switched. Only frames to and from your own machine and
broadcast frames on the local subnetwork will be accessible to the analyzer. This is fine for
analyzing software you are using or developing but it gives you no way to observe traffic
between machines other than your own.

\begin{enumerate}

\item Start Wireshark, capture a few packets, and observe their decoding to familiarize yourself
  with how Wireshark works in general.

\item Using your Windows daytime client, connect to your Linux daytime server and use Wireshark
  to observe the session. Use display filtering so that only the packets involved in the daytime
  communication are shown.
  
\item Pick one of the decoded packets from above. What value does the time-to-live (TTL) field
  have in the IP header? Is the TTL different in the client's and the server's packets?

\end{enumerate}

\section{Subversion}

Subversion is an open source version control system. It is used to coordinate changes to a
program that are made by multiple developers simultaneously. All significant programming
projects use some kind of version control. In the second half of this course we will be working
on a home-grown IM system called ``Chatter.'' This system uses a distributed object framework
that is conceptually simple, yet surprisingly powerful. Chatter is stored in my personal
Subversion repository.

Using Subversion properly is somewhat complicated. Since it is team-oriented software, there is
a significant social aspect to using it. We will go over these issues later.

\subsection*{Procedure}

On the class web site there is a document describing how to use Subversion. In the lab we have
the TortoiseSVN client installed. Use it to check out the project at the URL

\vspace{1.0ex}
\centerline{\tt svn://whirlwind.cis.vtc.edu/Chatter}

to some suitable location in your home directory on the file server. By doing a checkout to your
home directory, your working copy of the project will be available to you anywhere you log in.

TortoiseSVN normally modifies a file's icons to reflect their status as versioned files.
However, by default it does not do this when the files are stored on a network drive. Probably
you want these special icons enabled. To activate them, right click on a folder and select the
TortoiseSVN option. Select ``Settings'' and then look for ``Icon Overlays'' under ``Look and
Feel.'' Check the box for network drives. You may have to do this whenever you log into a new
machine for the first time.

Feel free to checkout a working copy of Chatter on \texttt{lemuria} as well. You will have to
use the command line Subversion client \texttt{svn} on the Linux hosts. Type \texttt{svn help}
for the syntax of the various command options.

\section{Optional Tools}

Some of you may wish to use alternate programming tools for certain assignments in this class.
In particular, you may wish to use the Open Watcom compilers for your C and C++ programming. I
will fully support the use of Open Watcom to the greatest extent feasible. However, be aware
that certain programs that we will be writing later in the class require libraries that are only
available for Visual C++ (and Java) so you won't be able to use Open Watcom for those
assignments. However, Open Watcom should be fine for most of the required Windows programming.
On the class web site you will find a page describing how to set up a project using the Open
Watcom IDE.

If you are interested in using Java or Scala for some of your programming, you should select a
Java/Scala development environment. I recommend Eclipse with the Scala plug-in (if you plan to
use Scala). Both Eclipse and the Scala plug-in are installed on all lab machines. On the class
web site you will find a page describing how to set up a Java project using Eclipse.

\section{Report}

For this class I will require you to use the \LaTeX\ typesetting system to write your lab
reports. Accordingly you should write a short report for this lab using \LaTeX\ so you can get
used to using that system. The actual content of the report is not relevant (you can just say,
``It was a really great lab''). The purpose of the report, like the purpose of the lab, is to
just help you get familiar with the tools you will be using later.

\LaTeX is installed on all lab machines so if you want you can prepare your report during the
lab time as well.

\end{document}

