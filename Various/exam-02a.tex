%%%%%%%%%%%%%%%%%%%%%%%%%%%%%%%%%%%%%%%%%%%%%%%%%%%%%%%%%%%%%%%%%%%%%%%%%%%%
% FILE         : exam-02.tex
% AUTHOR       : (C) Copyright 2010 by Peter C. Chapin <PChapin@vtc.vsc.edu>
% LAST REVISED : 2010-05-03
% SUBJECT      : Exam #2 for CIS-3152
%
%%%%%%%%%%%%%%%%%%%%%%%%%%%%%%%%%%%%%%%%%%%%%%%%%%%%%%%%%%%%%%%%%%%%%%%%%%%%

\documentclass[12pt]{examdesign}

% Options.
\Fullpages
% \NoRearrange
% \NoKey
\NumberOfVersions{2}

% Class and exam information.
\class{CIS-3152: Networks II}
\examname{Exam \#2}

% This setting seems to be ignored (in the answer key at least). \setlength{\parskip}{1.75ex
%   plus0.5ex minus0.5ex}

\begin{document}

% I can't seem to be able to use namedata@vspace below (see documen- tation). I can use the
% default \namedata but I can't seem to redefine \namedata to suit my needs.
%
\begin{examtop}
  \parbox{3in}{\classdata \\
               \examtype, Form: \fbox{\textsf{\Alph{version}}}}
  \hfill
  \parbox{3in}{\normalsize Name: \hrulefill \\[2.0ex]
                           Date: \hrulefill }
  \bigskip
\end{examtop}

% This exam covers the following topics. Since I am scrambling the sections, I can build the
% exam with questions for the same topic together. That makes it easier to verify that all
% topics are being covered and to check the balance of questions between the various topics.
%
% SMTP
% RFC-2822, MIME
% Character sets (Unicode)
% XML, XML schema
% RPC

\begin{truefalse}
  
  Answer each of the following questions ``True'' or ``False''. Each question is worth two
  points. If you decide to change your answer to any question, draw an `X' through your old
  answer and write your new answer next to it. Do not try to change a `T' to an `F' or
  visa-versa!

  % 1 (SMTP)
  \begin{question}
    \answer{True} Legacy SMTP (that is, SMTP with no extensions) is a stop-and-wait protocol.
  \end{question}
  
  % 2 (SMTP)
  \begin{question}
    \answer{False} SMTP requires that a message be delivered to a single recipient. If multiple
    recipients are intended, the message must be transmitted multiple times.
  \end{question}

  % 3 (RFC-2822, MIME)
  \begin{question}
    \answer{False} In an RFC-2822 formatted message, each header line must be indented.
  \end{question}

  % 4 (RFC-2822, MIME)
  \begin{question}
    \answer{True} If an old mail program (without MIME support) received a MIME formatted email
    message it should still be able to properly display the sender's name and address as well as
    the message's subject line and date stamp.
  \end{question}

  % 5 (Character sets, Unicode)
  \begin{question}
    \answer{False} In the UTF-16 encoding of Unicode, every possible character occupies exactly
    16 bits.
  \end{question}
  
  % 6 (Character sets, Unicode)
  \begin{question}
    \answer{True} In Unicode a ``combining character'' is a character that in some way modifies
    the meaning or appearance of the character occuring before it in the character stream.
  \end{question}

  % 7 (XML, XML Schema)

  \begin{question}
    \answer{True} The purpose of XML is to allow the structure of complex documents to be made
    explicit so that programs can read that structure more easily.
  \end{question}

  % 8 (XML, XML Schema)
  \begin{question}
    \answer{False} An XML Schema describes how the various elements in a document are to be
    formatted for display.
  \end{question}

  % 9 (RPC)
  \begin{question}
    \answer{False} The primary goal of any RPC system is to improve the performance (data
    transfer rate) of client/server communication.
  \end{question}
  
  % 10 (RPC)
  \begin{question}
    \answer{True} Sun's RPC can be used between computers using different processor
    architectures. For example a client on a 32 bit x86 machine can communicate with a server on
    a 64 bit Alpha based machine.
  \end{question}

\end{truefalse}

\begin{multiplechoice}
  
  Indicate the \emph{best} answer in each of the following questions by circling the letter of
  your choice. Each question is worth two points. If you decide to change your answer to any
  question, draw an 'X' through your old answer and circle your new answer.

  % 1 (SMTP)
  \begin{question}
    SMTP is called a ``store and forward'' protocol. This is because

    \choice{Extensions have been added to the SMTP protocol for specifying forwarding addresses.}
    \choice{Every SMTP server is required to save a copy of all messages that pass through it.}
    \choice{The SMTP protocol is a stop-and-wait protocol.}
    \choice[!]{An SMTP server can receive a message even if it is not the ultimate destination
      for that message.}
  \end{question}

  % 2 (SMTP)
  \begin{question}
    In theory SMTP is unaware of the content of the mail messages it transmits. However, this is
    not entirely true because

    \choice{SMTP servers examine the ``To'' header to determin the destination of the message.}
    \choice{SMTP servers verify the ``Date'' header and change it if it is inconsistent (for
      example if it contains a date in the future).}
    \choice{As a security measure, SMTP servers verify that the ``From'' header of the message
      agrees with the address given to them in the ``MAIL FROM'' SMTP command.}
    \choice[!]{Each SMTP server edits the message to add information about how the message was
      routed through that server.}
  \end{question}
  
  % 3 (RFC-2822, MIME)
  \begin{question}
    Correctly formed RFC-2822 messages are entirely composed of US-ASCII text. This requirement
    exists because

    \choice[!]{The SMTP protocol was originally defined as only able to transmit US-ASCII data.}
    \choice{Such messages are easier to read in various tools.}
    \choice{The MIME standard defines a way to convert binary data into US-ASCII text.}
    \choice{The first email applications were developed in the United States where English is
      the primary language.}
  \end{question}

  % 4 (RFC-2822, MIME)
  \begin{question}
    Base 64 encoding is useful because

    \choice{It is faster to process on 64 bit machines.}
    \choice[!]{It can transform binary data into plain ASCII text.}
    \choice{The encoded output is still nearly readible text if the input was mostly US-ASCII
      already.}
    \choice{The encoded output is essentially the same size as the input.}
  \end{question}

  % 5 (Character sets, Unicode)
  \begin{question}
    Some character encodings have a special ``shift'' character. After seeing the shift
    character, subsequent bytes are interepreted differently until a new shift back to the
    initial state is seen. The \emph{advantage} of this scheme over, say, UTF-16 encoded Unicode
    is

    \choice[!]{With multiple shift states a large number of characters can be encoded using the
      same byte and, if shifts occur infrequently, this results in a compact encoding of the
      text.}
    \choice{It is easy to randomly access a string to, for example, display a single character
      from the middle of the string.}
    \choice{You can count the number of characters in a string by simply counting the number of
      bytes in the string.}
    \choice{It is not necessary to define combining characters to handle accents and similar
      things.}
  \end{question}
  
  % 6 (Character sets, Unicode)
  \begin{question}
    Unicode was developed to solve which of the following problems:

    \choice{Creating documents containing a mixture of languages.}
    \choice{Incompatible character sets causing complications when sharing data.}
    \choice{Lack of support for obscure languages and writing systems.}
    \choice[!]{All of the above}
  \end{question}

  % 7 (XML)
  \begin{question}
    An XML file is said to be ``valid'' if

    \choice{It is well formed.}
    \choice{All elements have both start and end tags that nest properly.}
    \choice{It starts with a proper XML declaration.}
    \choice[!]{It is both well formed and obeys the constraints specified in a particular schema.}
  \end{question}

  % 8 (XML)
  \begin{question}
    The W3C's XML Schema language is itself an XML vocabulary. This is good because

    \choice{It is a compact way to represent schema information.}
    \choice{XML is a standard.}
    \choice[!]{XML schemas can be manipulated using generic XML tools.}
    \choice{People who work with XML have less to learn and remember.}
  \end{question}

  % 9 (RPC)
  \begin{question}
    Many RPC systems, including Sun RPC, define an ``interface definition language'' of some
    kind that must be used to express the interface to remotely callable services. What is the
    main advantage of doing this?

    \choice{It improves the execution performance of the system.}
    \choice[!]{It allows the interface to be described abstractly without concern about the
      specifics of the underlying programming environment.}
    \choice{It allows different RPC systems to interoperate.}
    \choice{It makes it possible for a service to be offered both remotely and locally as an
      ordinary library.}
  \end{question}
  
  % 10 (RPC)
  \begin{question}
    What is the purpose of the XDR standard?

    \choice{It is used to check argument and return types in remote calls.}
    \choice{It is used by Sun RPC for compatibility with an earlier (now defunct) standard.}
    \choice[!]{It defines a cross-platform way of encoding binary data.}
    \choice{It defines a set of minimum requirements on all implementations of Sun RPC.}
  \end{question}
  
\end{multiplechoice}
\pagebreak

\begin{shortanswer}
  \begin{question}
    (10 pts) To use XML it is not necessary to define a schema. Users of an XML language can
    just agree ahead of time about the elements to be used and any restrictions that apply to
    their use. What advantage(s) is(are) there to defining and using a schema?

    \begin{answer}
      A schema formally describes the XML language in a precise, unambiguous manner.
      Furthermore, conformance to the schema by an instance document can be machine checked. In
      cases where those sharing XML documents have little or no prior communication, a schema
      can increase the robustness and reliability of the information exchange.

      In addition, a schema clarifies what error handling must be done by applications and what
      facilities must be supported by, for example, style sheets.
    \end{answer}
  \end{question}
\end{shortanswer}

%\begin{shortanswer}
%  \begin{question}
%    (10 pts) What are the pros and cons of UTF-8 vs UTF-16 encoding for Unicode characters?
%
%    \begin{answer}
%      If the document contains mostly English, or any other language that uses the ASCII
%      character set, each character in the document will map to a single byte under UTF-8
%      encoding. However a UTF-16 encoding of the same text would be twice as large and hence
%      UTF-8 saves space in this case.
%      
%      However, if the document contains a large number of non-ASCII characters, many (most) of
%      those characters would map to three bytes using UTF-8 encoding. In this case UTF-8 is
%      actually less space efficient than UTF-16. In addition, since UTF-8 is a variable width
%      encoding its use with non-ASCII characters creates complications for random character
%      access, measuring the length of a string, and so forth.
%      
%      To summarize: UTF-8 is an excellent choice for documents that are mostly composed of ASCII
%      characters, but a poor choice for documents that are mostly non-ASCII.
%    \end{answer}
%  \end{question}
%\end{shortanswer}

\end{document}

