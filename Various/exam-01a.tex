%%%%%%%%%%%%%%%%%%%%%%%%%%%%%%%%%%%%%%%%%%%%%%%%%%%%%%%%%%%%%%%%%%%%%%%%%%%%
% FILE         : exam-01a.tex
% AUTHOR       : (C) Copyright 2010 by Peter C. Chapin
% LAST REVISED : 2010-03-12
% SUBJECT      : Exam #1 for CIS-3152
%
%%%%%%%%%%%%%%%%%%%%%%%%%%%%%%%%%%%%%%%%%%%%%%%%%%%%%%%%%%%%%%%%%%%%%%%%%%%%

\documentclass[12pt]{examdesign}

% Options.
\Fullpages
\NoRearrange
% \NoKey
\NumberOfVersions{1}

% Class and exam information.
\class{CIS--3152}
\examname{Exam \#1}

% This setting seems to be ignored (in the answer key at least). \setlength{\parskip}{1.75ex
%   plus0.5ex minus0.5ex}

\begin{document}

% I can't seem to be able to use namedata@vspace below (see documen- tation). I can use the
% default \namedata but I can't seem to redefine \namedata to suite my needs.
%
\begin{examtop}
  \parbox{3in}{\classdata \\
               \examtype, Form: \fbox{\textsf{\Alph{version}}}}
  \hfill
  \parbox{3in}{\normalsize Name: \hrulefill \\[2.0ex]
                           Date: \hrulefill }
  \bigskip
\end{examtop}

\begin{shortanswer}
  \begin{question}
    What is the advantage of using a concurrent TCP server? On the other hand, under what
    conditions would it be appropriate to use an iterative TCP server?

    \begin{answer}
      A concurrent server is able to service multiple connections at the same time. This is
      essential in cases where the time required to service a connection is long or outside the
      control of the server. For example if the client intends to download a large file or if
      the server is concerned that a client might connect and then wait indefinitely before
      issuing any commands, the server should be a concurrent server.

      An iterative server is appropriate in cases where the time required to service a
      connection is short and totally under the control of the server. For example in the
      daytime protocol, as soon as the client connects, the server sends a short string and then
      does the active close. There is nothing the client can do to extend the connection time.
      An iterative server works in this case and is easier to write.
    \end{answer}
  \end{question}
  
\pagebreak

  \begin{question}
    Suppose you use netstat to observe a TCP connection on your machine and find it in the
    SYN\_SENT state. What does this tell you? Observing a connection in the SYN\_SENT state is
    unusual. Why don't you normally see it?

    \begin{answer}
      The SYN\_SENT state occurs just after TCP sends its initial SYN segment and before it has
      received the corresponding SYN/ACK segment. This happens at the very beginning of a
      connection. If a connection is stuck in this state it means that no SYN/ACK has been
      received from the other side. Most likely the remote machine does not exist or is
      unavailable. You don't normally see connections in SYN\_SENT because under ordinary
      conditions the time spent in that state is very small.
      
      Note that if the remote machine was reachable but just was not listening on the
      destination port, the remote machine would have sent an ICMP ``Port unreachable'' message
      and the connection would have been aborted at once. Thus you would not expect to see the
      connection in the SYN\_SENT state under these circumstances.
    \end{answer}
  \end{question}

\pagebreak

  \begin{question}
    What does it mean to say that TCP is a ``windowed'' protocol? What is the difference between
    the congestion window and the receiver window?
    
    \begin{answer}
      TCP is a windowed protocol in contrast to a ``stop and wait'' protocol. In the stop and
      wait case, the sender would wait for an acknowledgment of each segment sent before sending
      the next segment. In cases where the round trip time between sender and receiver is long,
      this results in very slow transfer rates. Instead TCP defines a ``window'' the size of
      which defines how much data can be in transit on the network at any given time. Under
      ideal conditions, this allows the sender to send continuously; each acknowledgment
      releasing some space in the window for the next segment to send.
      
      TCP actually defines two windows: the congestion window and the receiver window. The
      congestion window reflects the sender's estimate of how much material the network can
      ``hold.'' The receiver window reflect's the amount of space left in the receiver buffer.
      By using the minimum of these two window sizes, the sender tries to avoid overload either
      the network or the receiver, whichever is more limited.
    \end{answer}
  \end{question}

\pagebreak

  \begin{question}
    TCP's congestion avoidance algorithm requires reasonably accurate round trip times in order
    to estimate an appropriate retransmission timeout interval. What happens if TCP thinks the
    round trip time is longer than it actually is? What happens if TCP thinks the round trip
    time is shorter than it actually is?

    \begin{answer}
      If the estimated round trip time (RTT) is too large, TCP will wait too long before timing
      out lost segments. The result is inefficient data transfer. If the estimated RTT is too
      small, TCP will timeout segments prematurely and make unnecessary retransmissions, thus
      adding to network congestion.
    \end{answer}
  \end{question}

\pagebreak

  \begin{question}
    A program that uses UDP will use the \texttt{sendto()} and \texttt{recvfrom()} functions
    instead of the \texttt{send()}/\texttt{recv()} or \texttt{write()}/\texttt{read()} functions
    that a TCP based program would use. The \texttt{sendto()} and \texttt{recvfrom()} functions
    both take a pointer to a sockaddr structure as a parameter. What is the purpose of this
    parameter and why do the UDP functions require it while the TCP functions do not?

    \begin{answer}
      The address of the remote host is provided to (or from) TCP when the connection is
      established (during \texttt{connect()} for the client or \texttt{accept()} for the
      server). Since that address remains the same for the duration of the connection there is
      no need to provide it again during the read/write operations.

      In contrast UDP does not have connections. Thus the address of the remote host must be
      provided to (or obtained from ) each read/write operation separately. This is the purpose
      of the sockaddr structure mentioned above.
    \end{answer}
  \end{question}

\end{shortanswer}
\end{document}
