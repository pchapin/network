%%%%%%%%%%%%%%%%%%%%%%%%%%%%%%%%%%%%%%%%%%%%%%%%%%%%%%%%%%%%%%%%%%%%%%%%%%%
% FILE    : lab07.tex
% AUTHOR  : (C) Copyright 2012 by Peter C. Chapin
% SUBJECT : MIME Lab for CIS-3152
%
% In this lab the students will build simple MIME messages and try sending them to various mail
% programs.
%%%%%%%%%%%%%%%%%%%%%%%%%%%%%%%%%%%%%%%%%%%%%%%%%%%%%%%%%%%%%%%%%%%%%%%%%%%

% ++++++++++++++++++++++++++++++++
% Preamble and global declarations
% ++++++++++++++++++++++++++++++++
\documentclass[twocolumn]{article}

%\pagestyle{headings}
%\setlength{\parindent}{0em}
%\setlength{\parskip}{1.75ex plus0.5ex minus0.5ex}

% +++++++++++++++++++
% The document itself
% +++++++++++++++++++
\begin{document}

% ----------------------
% Title page information
% ----------------------
\title{CIS--3152 Lab \#7\\MIME}
\author{Peter C. Chapin\thanks{PChapin@vtc.vsc.edu}\\
  Vermont Technical College}
\date{Last Revised: March 18, 2012}
\maketitle

\section*{Introduction}

In this lab you will manually create several MIME messages and mail them to some mail user
agents to see how they are handled. By creating the messages manually you will bypass any
problems you might have due to limited sending mail user agents. We will be experimenting with a
few features that are not widely implemented.

The MIME standard is defined by RFC-2045 and RFC-2046. Note especially RFC-2046 where the
various media types are defined.

\section{Procedure}

Due to restrictions placed on the lab machines (or by your personal malware protection software)
you may have difficulty sending email to remote machines from your hackbox session. To get
around this you can send mail to the student user on hackbox instead. Unfortunately this will
limit your options regarding mail user agents. You can use either Evolution or Alpine on the
hackbox system.

To bypass the sending mail user agent you can execute the \texttt{sendmail} program
directly\footnote{On hackbox \texttt{sendmail} is just a compatibility wrapper around the
  Postfix mail transport agent; it is not the actual \texttt{sendmail} MTA}. Note that
\texttt{sendmail} is not in the default path so you will need to specify a path for it
explicitly when you run it. The syntax to use is

\begin{verbatim}
$ /usr/sbin/sendmail \
   student@hackbox.vtc.edu < message.txt
\end{verbatim}

Here `student@hackbox.vtc.edu' is the address to which you are sending the mail. The file
\texttt{message.txt} should contain an RFC-5322 formatted message (perhaps with MIME features).
Notice that \texttt{sendmail} expects to find the message at its standard input and not named as
a command line argument.

The \texttt{sendmail} program will automatically add a Date header. Although the header is
required by RFC-5322, you do not need to create it. The Date header has a fussy syntax that must
be followed closely; it's nice that you don't have to worry about it.

Ideally you would use at least two different email programs to receive your test messages. Be
sure to record which mail clients you use and their versions (if possible) in your report.

For some of the exercises you will need to produce base64 encoded data. You can use the
\texttt{base64} program for this purpose. Read the manual page for it on hackbox.

Follow these steps.

\begin{enumerate}

\item Start by creating a simple ``Hello, World'' email message and verifying that you can send
  it to yourself.

\item Try a text/plain message that has been base64 encoded. In other words, encode the main
  part of the message. Note that spammers used to try doing this to avoid spam filters since in
  the past spam filters didn't try decoding such messages to examine them. Modern spam filters
  are not that dumb.

\item Try a multipart/mixed message where the second part is the base64 encoding of some binary
  file (for example, a small image). The encoded file should appear as an attachment in your
  receiving mail program(s). Be sure to use an appropriate MIME type for the attachment or else
  the receiving mail program won't know what to do with it.

\item Try a multipart/mixed message where the second part is also a multipart message consisting
  of a text/plain part and a base64 encoded binary file. OPTIONAL: It would be interesting to
  build a sequence of messages with ever increasing depth of nesting to see how many nesting
  levels of multipart messages your mail user agents can handle.

\item Try a multipart/alternative message consisting of two text/plain parts that are very
  different from each other. Although this is not the intent of multipart/alternative, it will
  allow you to clearly see which alternative your mail user agents are selecting.

\item Try one of the following two options.
  \begin{enumerate}
  \item Try a multipart/digest message. Such messages are similar to multipart/mixed except each
    part has type message/rfc822 (See RFC-2046). Do your mail user agents handle message digests
    any differently from ordinary multipart/mixed messages?

  \item Try a message/partial message. In this case a single message is broken into fragments
    and sent in several (smaller) parts. Mail user agents are supposed to reassemble the parts
    automatically. Does this happen? See RFC-2046 for the details on constructing
    message/partial messages.
  \end{enumerate}
\end{enumerate}

\section{Report}

Submit a report showing the messages you constructed for each part above and a few paragraphs
summarizing how they worked. What conclusion can you draw?

\end{document}
