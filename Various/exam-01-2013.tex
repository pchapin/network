%%%%%%%%%%%%%%%%%%%%%%%%%%%%%%%%%%%%%%%%%%%%%%%%%%%%%%%%%%%%%%%%%%%%%%%%%%%%
% FILE         : exam-01.tex
% AUTHOR       : (C) Copyright 2013 by Peter C. Chapin
% LAST REVISED : 2013-03-17
% SUBJECT      : Exam #1 for CIS-3152
%
% Send comments or bug reports to:
%
%       Peter C. Chapin
%       Computer Information Systems
%       Vermont Technical College
%       Williston, VT 05495
%       PChapin@vtc.vsc.edu
%%%%%%%%%%%%%%%%%%%%%%%%%%%%%%%%%%%%%%%%%%%%%%%%%%%%%%%%%%%%%%%%%%%%%%%%%%%%

\documentclass[12pt]{examdesign}

% Options.
\Fullpages
\NoRearrange
% \NoKey
\NumberOfVersions{1}

% Class and exam information.
\class{CIS-3152}
\examname{Exam \#1}

% This setting seems to be ignored (in the answer key at least).
% \setlength{\parskip}{1.75ex plus0.5ex minus0.5ex}

\begin{document}

% I can't seem to be able to use namedata@vspace below (see documen-
% tation). I can use the default \namedata but I can't seem to redefine
% \namedata to suite my needs.
%
\begin{examtop}
  \parbox{3in}{\classdata \\
               \examtype, Form: \fbox{\textsf{\Alph{version}}}}
  \hfill
  \parbox{3in}{\normalsize Name: \hrulefill \\[2.0ex]
                           Date: \hrulefill }
  \bigskip
\end{examtop}

\textit{This is a take-home exam. Do not work with anyone else on the exam or post questions
  about the exam to any Internet forums, etc. You can use any other resources including the
  text, your notes, or documents you find online (including old forum discussions started by
  others). Please include a URL for any Internet resources you use. If you have any questions
  about the exam, feel free to contact me.}

\begin{shortanswer}

  % Sockets programming.
  \begin{question}
    The call to \texttt{connect()} in a typical sockets based TCP program uses a second argument
    that looks like \texttt{(struct sockaddr *) \&server\_address} where
    \texttt{server\_address} is a \texttt{struct sockaddr\_in}. Why is the complicated type
    conversion necessary? In other words: why wasn't \texttt{connect()} declared to take a
    \texttt{struct sockaddr\_in *} in the first place?

    \begin{answer}
      The sockets API was designed to work with many different protocol families. Each protocol
      family has its own way of addressing nodes and thus has its own socket address structure.
      The \texttt{connect()} function must be able to take a pointer to any of these structure
      types. To deal with that in C, \texttt{connect()} has been declared as taking a pointer to
      a generic structure with the understanding that the caller can send in a pointer to any of
      the supported structures (by using an appropriate type cast).
    \end{answer}
  \end{question}

%\pagebreak

  % Iterative vs concurrent servers.
  \begin{question}
    What is the advantage of using a concurrent TCP server? On the other hand, under what
    conditions would it be appropriate to use an iterative TCP server?

    \begin{answer}
      A concurrent server is able to service multiple connections at the same time. This is
      essential in cases where the time required to service a connection is long or outside the
      control of the server. For example if the client intends to download a large file or if
      the server is concerned that a client might connect and then wait indefinitely before
      issuing any commands, the server should be a concurrent server.

      An iterative server is appropriate in cases where the time required to service a
      connection is short and totally under the control of the server. For example in the
      daytime protocol, as soon as the client connects, the server sends a short string and then
      does the active close. There is nothing the client can do to extend the connection time.
      An iterative server works in this case and is easier to write.
    \end{answer}
  \end{question}

%\pagebreak

  % TCP state diagram.
  \begin{question}
  	Answer two of the three questions below.
    \begin{enumerate}
    \item When a client program calls \texttt{connect()}, what state or states in the TCP state
      diagram does the connection touch?\footnote{By ``touch'' I mean ``pass through'' or ``end
        up in.''} You can assume normal, error free behavior.
    \item What is the difference between ``active close'' and ``passive close?''
    \item What sequence of states does the TCP connection touch when the side doing the passive
      close calls \texttt{close()}? You can assume normal, error free behavior.
    \end{enumerate}
    
    \begin{answer}
    \begin{enumerate}
    
    \item When calling \texttt{connect()} the client passes through the SYN\_SENT state and ends
      up in the ESTABLISHED state.
    \item The side that does the active close is the side that initially calls \texttt{close()}
      to end a connection. This side sends the first FIN segment. The side that does the passive
      close acknowledges the initial FIN segment. It then later calls \texttt{close()} as well
      to send a FIN segment of its own.
    \item When the passive side calls \texttt{close()} it starts in CLOSE\_WAIT and passes
    through LAST\_ACK.
    \end{enumerate}
    \end{answer}
  \end{question}

%\pagebreak

  % UDP
  \begin{question}
    UDP is an unreliable protocol. However, some services are most appropriately implemented as
    UDP services rather than using the more reliable TCP (DNS and TFTP are like this). Give at
    least one \emph{advantage} that UDP has over TCP in these cases. Under what conditions is
    the unreliability of UDP not an issue?

    \begin{answer}
      UDP's advantage over TCP is its lower overhead. This is particularly significant if the
      messages being exchanged are small. UDP works well when a client's request and the
      server's reply both fit into a single datagram, and when the service being provided has
      the property of being idempotent. UDP is also a good choice in cases where its
      unreliability is not an issue such as streaming media (where lost data degrades the
      quality of the service but does not cause an outright failure of the service).
    \end{answer}
  \end{question}

%\pagebreak

  % UDP Programming
  \begin{question}
    A program that uses UDP will use the \texttt{sendto()} and \texttt{recvfrom()} functions
    instead of the \texttt{send()}/\texttt{recv()} or \texttt{write()}/\texttt{read()} functions
    that a TCP based program would use. The \texttt{sendto()} and \texttt{recvfrom()} functions
    both take a pointer to a sockaddr structure as a parameter. What is the purpose of this
    parameter and why do the UDP functions require it while the TCP functions do not?

    \begin{answer}
      The address of the remote host is provided to (or from) TCP when the connection is
      established (during \texttt{connect()} for the client or \texttt{accept()} for the
      server). Since that address remains the same for the duration of the connection there is
      no need to provide it again during the read/write operations.

      In contrast UDP does not have connections. Thus the address of the remote host must be
      provided to (or obtained from ) each read/write operation separately. This is the purpose
      of the sockaddr structure mentioned above.
    \end{answer}
  \end{question}

\end{shortanswer}
\end{document}
