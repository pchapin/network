%%%%%%%%%%%%%%%%%%%%%%%%%%%%%%%%%%%%%%%%%%%%%%%%%%%%%%%%%%%%%%%%%%%%%%%%%%%
% FILE         : lab07.tex
% AUTHOR       : (C) Copyright 2001 by Peter Chapin
% LAST REVISED : March 5, 2001
% SUBJECT      : HTTP
%
% In this lab the students will write a short utility that serves as an
% HTTP client.
%
% Send comments to Peter Chapin (pchapin@vtc.vsc.edu) or at the snail mail
% address of:
%
%       Peter Chapin
%       Vermont Technical College
%       Main Street
%       Randolph Center, VT. 05061
%%%%%%%%%%%%%%%%%%%%%%%%%%%%%%%%%%%%%%%%%%%%%%%%%%%%%%%%%%%%%%%%%%%%%%%%%%%

% ++++++++++++++++++++++++++++++++
% Preamble and global declarations
% ++++++++++++++++++++++++++++++++
\documentclass{article}
% \pagestyle{headings}
\setlength{\parindent}{0em}
\setlength{\parskip}{1.75ex plus0.5ex minus0.5ex}

% +++++++++++++++++++
% The document itself
% +++++++++++++++++++
\begin{document}

% ----------------------
% Title page information
% ----------------------
\title{CS-301 Lab \#7\\HTTP}
\author{\copyright Copyright 2001 by Peter Chapin}
\date{Last Revised: March 6, 2001}
\maketitle

\section{Introduction}

This lab has yet to be written. In the first edition of this course I
just told them what to do in class... I had them observe some HTTP
transactions with the network analyzer and then write a short client (in
Java) that fetched a document with HTTP and saved it to disk.

\section{Report}

Turn in a listing of your program and answer the following questions.

\begin{enumerate}

\item Question 1.

\end{enumerate}

\end{document}
