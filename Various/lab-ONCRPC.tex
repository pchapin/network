%%%%%%%%%%%%%%%%%%%%%%%%%%%%%%%%%%%%%%%%%%%%%%%%%%%%%%%%%%%%%%%%%%%%%%%%%%%
% FILE    : lab08.tex
% AUTHOR  : (C) Copyright 2012 by Peter C. Chapin <PChapin@vtc.vsc.edu>
% SUBJECT : RPC
%
% In this lab the students will experiment with an RPC client/server.
%%%%%%%%%%%%%%%%%%%%%%%%%%%%%%%%%%%%%%%%%%%%%%%%%%%%%%%%%%%%%%%%%%%%%%%%%%%

% ++++++++++++++++++++++++++++++++
% Preamble and global declarations
% ++++++++++++++++++++++++++++++++
\documentclass[twocolumn]{article}

%\pagestyle{headings}
%\setlength{\parindent}{0em}
%\setlength{\parskip}{1.75ex plus0.5ex minus0.5ex}

% +++++++++++++++++++
% The document itself
% +++++++++++++++++++
\begin{document}

% ----------------------
% Title page information
% ----------------------
\title{CIS--3152 Lab \#8\\RPC File Services}
\author{Peter C. Chapin\thanks{PChapin@vtc.vsc.edu}\\
  Vermont Technical College}
\date{Last Revised: March 22, 2012}
\maketitle

\section*{Introduction}

In this lab you will investigate the action of ONC RPC (Sun RPC). You will write a simple server
consisting of several short procedures as well as a client that makes use of those procedures.
You will then demonstrate how the client can call the procedures in the server and, if the
network topology permits, you will look at the conversation between the client and server using
the network analyzer.

\section{The Procedures}

Since the point of this lab is to observe RPC, we will make the procedures themselves fairly
simple. In a ``real'' application the procedures offered by the server would most likely be
fairly complex; implementing them would be the bulk of the work. Nevertheless the procedures
described here will illustrate a number of interesting points about RPC.

You are to write a simple file access server. The procedures are as follows:

\begin{verbatim}
typedef string filename_t<256>;

struct readresult {
  int    count;
  opaque buffer[512];
};

bool       OPEN(filename_t) = 1;
readresult READ(int)        = 2;
void       CLOSE(void)      = 3;

\end{verbatim}

These procedures allow a client to open a file by name with OPEN, read it using READ, and close
it with CLOSE. Only one file at a time can be open. If an attempt is made to open a file while
another file is already open, the attempt will fail.

The parameter to READ specifies how many bytes are requested (with a limit of 512). If the count
member of the readresult returned by READ is zero, then the file has reached the end. Notice
that this interface gives you no way to scan the server's directory and find the name of the
files there. You must know the file name ahead of time.

Put the interface definition into \texttt{file\_services.x}, along with information about the
program number and version number that you want to use, and compile it with \texttt{rpcgen}.
Then implement the server side so that it only allows files from a particular, special directory
to be accessed. Write a client program that accepts a file name and then displays the contents
of that file on the terminal.

When you are done creating these programs you will have implemented a simple file access
protocol using RPC.

\section{Observations}

Try using a network analyzer to decode the traffic between the client and server. What
additional information is placed in the RPC packets besides your procedure parameters and return
values?

Look at the files generated by \texttt{rpcgen}. Can you explain how the client stubs work? What
does the server main program do? How is the server stub implemented?

\section{Report}

Turn in your \texttt{file\_services.x} file as well as your client and server source files. You
do not need to turn in any of the files created by rpcgen. Write a report that describes how
your system works. Consider the following questions.

\begin{enumerate}

\item Which one of your procedures, if any, are idempotent? Which ones have side effects?

\item What would happen if the client program terminated unexpectedly (crashed) while it was in
  the middle of reading a file from the server? Would the server be able to recover?

\item Estimate the maximum rate at which this system could read file data (for example, in the
  case of a very large file). Assume the network operates at 100 Mbps, the underlying protocol
  is UDP, and the one way transit time is 0.5ms. Assume also that the file access and processing
  times on either end of the RPC connection are negligable.

\item The READ procedure returns a 512 byte array (along with a count). A more flexible
  arrangement would be for READ to return a variable length array of bytes. (\texttt{opaque
    buffer<>}). If that were done would there be any issues with returning, for example, a 256
  KByte chunk of the file? HINT: Think about the transport protocol used to support the RPC
  transaction.

\end{enumerate}

\end{document}
