%%%%%%%%%%%%%%%%%%%%%%%%%%%%%%%%%%%%%%%%%%%%%%%%%%%%%%%%%%%%%%%%%%%%%%%%%%%%
% FILE         : final.tex
% AUTHOR       : (C) Copyright 2008 by Peter C. Chapin
% LAST REVISED : 2008-05-03
% SUBJECT      : Final for CIS-3152
%
% Send comments or bug reports to:
%
%       Peter C. Chapin
%       Electrical and Computer Engineering Technology
%       Vermont Technical College
%       Randolph Center, VT 05061
%       Peter.Chapin@vtc.vsc.edu
%%%%%%%%%%%%%%%%%%%%%%%%%%%%%%%%%%%%%%%%%%%%%%%%%%%%%%%%%%%%%%%%%%%%%%%%%%%%

\documentclass[12pt]{examdesign}

\usepackage{newvbtm}

% Options.
\Fullpages
\NoRearrange
% \NoKey
\NumberOfVersions{1}

% Class and exam information.
\class{CIS-3152}
\examname{Final}

% This setting seems to be ignored (in the answer key at least).
% \setlength{\parskip}{1.75ex plus0.5ex minus0.5ex}

\newverbatim{codelisting}{}{}{}{}

\newsavebox{\udpQuestion}
\newsavebox{\xmlQuestion}

\begin{document}

\begin{lrbox}{\udpQuestion}%
\begin{minipage}{0.9\textwidth}
\begin{codelisting}
// Receive a packet from the server.
address_size = sizeof(struct sockaddr_in);
recv_count = recvfrom(
  socket_handle,
  data_packet,
  DATA_LENGTH,
  0,
  (struct sockaddr *)&incoming_address,
  &address_size
);
\end{codelisting}%
\end{minipage}%
\end{lrbox}

\begin{lrbox}{\xmlQuestion}%
\begin{minipage}{0.9\textwidth}
\begin{codelisting}
<xs:element name="recipe">
  <xs:complexType>
    <xs:sequence>
      <xs:element name="ingredient" type="xs:string"
                  minOccurs="1" maxOccurs="unbounded"/>
      <xs:element name="procedure" type="xs:string"/>
    </xs:sequence>
    <xs:attribute name="type" type="xs:string" use="required"/>
  </xs:complexType>
</xs:element>
\end{codelisting}%
\end{minipage}%
\end{lrbox}

\begin{examtop}
  \parbox{3in}{\classdata \\
               \examtype, Form: \fbox{\textsf{\Alph{version}}}}
  \hfill
  \parbox{3in}{\normalsize Name: \hrulefill \\[2.0ex]
                           Date: \hrulefill }
  \bigskip

\end{examtop}

\begin{truefalse}[title={True/False}]
  
Answer each of the following questions `T' for true or `F' for false. If you want to change an
answer, write over the old answer with an `X' and then write the new answer beside the old
answer. Each question is worth two points.

%  % Client/server programming with sockets.
%  \begin{question}
%  \answer{F} When programming with TCP sockets, it is the client program that calls
%  \texttt{accept()} to create a new connection.
%  \end{question}

  % TCP protocol.
  \begin{question}
  \answer{F} TCP requires that both ends of the connection use the same initial sequence
  numbers.
  \end{question}
  
%  % UDP protocol.
%  \begin{question}
%  \answer{T} UDP based protocols are easier to implement if the service provided is idempotent.
%  \end{question}

  % Trivial FTP.
  \begin{question}
  \answer{T} The trivial FTP protocol as we implemented in lab does not use TCP.
  \end{question}
  
%  % IPv6.
%  \begin{question}
%  \answer{F} To use IPv6, the TCP protocol must be modified.
%  \end{question}

  % Character sets and Unicode.
  \begin{question}
  \answer{T} The UTF-8 encoding of Unicode is a variable length encoding.
  \end{question}

%  % Mail protocols.
%  \begin{question}
%  \answer{T} Legacy SMTP (that is, SMTP with no extensions) is a stop-and-wait protocol.
%  \end{question}

  % MIME.
  \begin{question}
  \answer{F} It is not possible to use base64 encoding for text/plain message bodies.
  \end{question}
  
%  % XML.
%  \begin{question}
%  \answer{T} An XML schema is a description of the allowed structure of specific kinds of XML
%  documents.
%  \end{question}

  % Middleware.
  \begin{question}
  \answer{T} The Ice middleware platform allows systems to be written in a variety of
  programming languages.
  \end{question}

\end{truefalse}

\begin{multiplechoice}[title={Multiple Choice}]
  
  Indicate the \emph{best} answer in each of the following questions by circling the appropriate
  letter. If you want to change an answer, mark over your old answer with an `X' and circle your
  new answer. Each question is worth two points.
  
%  % Client/server programming with sockets.
%  \begin{question}
%    A TCP connection is defined by
%    
%    \choice{The socket handle returned by \texttt{socket}.}
%    \choice[!]{The IP address and port number associated with each end point.}
%    \choice{The type of service that is accessed over the connection.}
%    \choice{The address given to \texttt{inet\_pton}.}
%  \end{question}

  % TCP protocol.
  \begin{question}
    When doing bulk data transfers, TCP is usually a better choice than UDP because

    \choice{There is less overhead in establishing the connection.}
    \choice{The packets are larger.}
    \choice{Each packet is acknowledged.}
    \choice[!]{It is a windowed protocol and hence able to use the network capacity more
    efficiently.}
  \end{question}

%  % UDP protocol.
%  \begin{question}
%    When programming with UDP it is important
%
%    \choice{To connect to the destination machine before trying to send any packets.}
%    \choice{To prepare the data as a stream to be sent in a single unit.}
%    \choice[!]{To arrange for all calls to \texttt{recvfrom} to time out if no packet is received.}
%    \choice{To avoid waiting for a reply packet before sending the next outgoing packet.}
%  \end{question}

  % Trivial FTP.
  \begin{question}
    A file transfer application is not normally a good application for UDP. However, the Trivial
    FTP protocol uses UDP anyway because

    \choice[!]{That way it can be implemented on a system with very few resources.}
    \choice{It is only used with very small files.}
    \choice{It does not need to be fast.}
    \choice{It is only used with files that do not need to be transfered with 100\% accuracy
    (such as multimedia files).}
  \end{question}

%  % IPv6.
%  \begin{question}
%    The purpose of an IPv6 configured tunnel is
%
%    \choice{To allow IPv6/IPv4 hosts a way of communicating with IPv4-only hosts.}
%    \choice{To allow IPv6/IPv4 hosts a way of communicating with each other over an IPv4
%    infrastructure.}
%    \choice[!]{To allow IPv6-only hosts a way of communicating with each other over an IPv4
%    infrastructure.}
%    \choice{To allow IPv6 routers a way of communicating with each other over an IPv4
%    infrastructure.}
%  \end{question}

  % Character sets and Unicode.
  \begin{question}
    The purpose of the byte order mark in UTF-16 is

    \choice{For compatibility with older character encodings.}
    \choice[!]{To give the receiving application a way of detecting the endian-ness of the
    character stream.}
    \choice{To let the recieving application know that UTF-16 is being used.}
    \choice{To indicate the start of Unicode text.}
  \end{question}

%  % Mail protocols.
%  \begin{question}
%    When the MIME standard was created, there was no need to change the SMTP protocol to support
%    MIME. The fundamental reason for this is
%
%    \choice{MIME was designed with SMTP's limitations in mind.}
%    \choice{MIME has nothing to do with SMTP. However, it was necessary to make some adjustments
%    to the format of email messages as described in the earlier version of RFC-2822.}
%    \choice[!]{MIME is higher in the protocol stack and thus MIME encoded messages are just
%    uninterpreted data to an SMTP server.}
%    \choice{SMTP servers had by that time been upgraded to support the more modern message
%    format described in RFC-2822.}
%  \end{question}

  % MIME.
  \begin{question}
    Quoted printable encoding is useful because

    \choice{It is faster to process on 64 bit machines.}
    \choice{It can transform binary data into plain ASCII text.}
    \choice[!]{The encoded output is still nearly readible text if the input was mostly US-ASCII
    already.}
    \choice{The encoded output is essentially the same size as the input.}
  \end{question}

%  % XML.
%  \begin{question}
%    XML documents can be tested for validity by generic validator programs (such as XSV). In an
%    application where XML is being used, what is the value in insisting that all documents are
%    valid?
%
%    \choice{It ensures that all documents can be properly styled to HTML (with the use of a
%    suitable XSLT style sheet).}
%    \choice{It ensures that all documents conform to the XML standard.}
%    \choice{It forces all documents to be syntactically correct and thus easier to read.}
%    \choice[!]{It simplifies error handling because valid documents must exhibit a specific
%    structure. Thus the application does not need to check for violations of that structure.}
%  
%  \end{question}

  % Middleware.
  \begin{question}
    In the Ice middleware system a ``proxy'' is

    \choice{An object that performs protocol translations on behalf of another object.}
    \choice[!]{A kind of pointer that refers to objects potentially located on remote machines.}
    \choice{An object implementing the \texttt{Ice::Proxy} interface.}
    \choice{A way of accessing the \texttt{Chatter} name service.}
  \end{question}

\end{multiplechoice}

\pagebreak

\begin{shortanswer}[title={Short Answer}]
  
  Answer three of the four following questions. \emph{Cross out the question you do not want
  graded!} Each question is worth 10 points.

  \begin{question}
    Consider the code fragment below. This is taken from a UDP client program. Explain the
    purpose of each of \texttt{recvfrom}'s arguments.

    \bigskip
    \usebox{\udpQuestion}
    \bigskip

    \begin{answer}
    \end{answer}
    \pagebreak
  \end{question}

  \begin{question}
    IPv6 addresses define three different ``scopes.'' What are those scopes and in what
    situations would each scope be used? Would it make sense for a host to have three different
    address (one in each scope) at once?

    \begin{answer}
    \end{answer}
    \pagebreak
  \end{question}

  \begin{question}
    Show the complete SMTP conversation you would expect to see when foo.com tries to send a
    mail message to bar.gov. Let the message be from satan@hell.net to god@heaven.org. For this
    question you can just indicate the message content itself as ``*message*'' in your answer.

    \begin{answer}
    \end{answer}
    \pagebreak
  \end{question}

%  \begin{question}
%    Show a MIME multipart/alternative message for the email described in the question above. Let
%    the first part be type text/plain and the second part be type text/html. Let the content be
%    ``Busy today? Want to do lunch?''
%
%    \begin{answer}
%    \end{answer}
%    \pagebreak
%  \end{question}

  \begin{question}
    Consider the fragment of an XML schema shown below. Show a corresponding fragment of an XML
    document that conforms to this schema.

    \bigskip
    \usebox{\xmlQuestion}
    \bigskip

    \begin{answer}
    \end{answer}
    \pagebreak
  \end{question}

%  \begin{question}
%    Suppose one defined an Ice interface to an email server as an alternative to SMTP for
%    sending mail. What advantages and disadvantages would such an interface have over the
%    traditional approach?
%
%    \begin{answer}
%    \end{answer}
%    \pagebreak
%  \end{question}

\end{shortanswer}

\end{document}
