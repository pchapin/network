%%%%%%%%%%%%%%%%%%%%%%%%%%%%%%%%%%%%%%%%%%%%%%%%%%%%%%%%%%%%%%%%%%%%%%%%%%%
% FILE         : lab09.tex
% AUTHOR       : (C) Copyright 2012 by Peter C. Chapin
% LAST REVISED : 2012-04-08
% SUBJECT      : Chatter
%
% In this lab the students will experiment with Ice.
%
%       Peter C. Chapin
%       Computer Information Systems Department
%       Vermont Technical College
%       Randolph Center, VT. 05061
%       PChapin@vtc.vsc.edu
%%%%%%%%%%%%%%%%%%%%%%%%%%%%%%%%%%%%%%%%%%%%%%%%%%%%%%%%%%%%%%%%%%%%%%%%%%%

% ++++++++++++++++++++++++++++++++
% Preamble and global declarations
% ++++++++++++++++++++++++++++++++
\documentclass[twocolumn]{article}
% \pagestyle{headings}
\setlength{\parindent}{0em}
\setlength{\parskip}{1.75ex plus0.5ex minus0.5ex}

% +++++++++++++++++++
% The document itself
% +++++++++++++++++++
\begin{document}

% ----------------------
% Title page information
% ----------------------
\title{CIS--3152 Lab \#9\\Chatter}
\author{\copyright\ Copyright 2012 by Peter C. Chapin}
\date{Last Revised: April 8, 2012}
\maketitle

\section*{Introduction}

In this lab you will work on a simple instant messaging system called \textit{Chatter}. In
Chatter each user runs a server that accepts lines of text for that user. In addition each user
runs a client to send text to other users. In our implementation the server and client are
independent processes.

In order for users to find each other on the network, the Chatter system makes use of a name
service. Each Chatter object connects to the name service and registers its proxy under the
user's nickname. When a client wants to chat with a particular user, the client program looks up
the user's proxy from the name service using the nickname as the key.

\section{Trying Chatter}

Your first step should be to try out the existing Chatter system. Checkout Chatter from its
Subversion repository using the Subversion URL of
\begin{verbatim}
svn://whirlwind.cis.vtc.edu/Chatter
\end{verbatim}

You can build Chatter on either a 64 bit Linux system using gcc (the Ice libraries are installed
on \texttt{lemuria}) or on 32 bit Windows using Microsoft Visual C++. Once you have checked out
the Chatter source, build the makefile in the root of the \texttt{Ice} folder to create the
entire system as it currently exists. Note that if you are using Windows you will need to use
Microsoft's \texttt{nmake} tool to do the build (not Visual Studio). For example in a ``Visual
Studio 2008 Command Prompt'' window execute
\begin{verbatim}
nmake /f Makefile.win32
\end{verbatim}

A Chatter name server is already running on the host \texttt{vortex.cis.vtc.edu} on the default
Chatter name service port of 9010 (this port value is hard coded into the various Chatter sample
programs). In addition a Chatter room server is also running on the network under the name
``VTC.'' Note that it has registered itself with the name server so the exact location where it
is running is not relevant.

Do the following to start experimenting with Chatter.

\begin{enumerate}

\item In one window bring up the Chatter ``server'' program. You will need to provide it with
  the nickname you would like to use on the network, the name of the host where the name server
  is running, and the TCP port you would like it to use. Messages to this name will be delivered
  to your server window. For example
\begin{verbatim}
server Peter vortex.cis.vtc.edu 9000
\end{verbatim}

  \emph{If you are using the lab Windows systems you may need to unblock the specified port on
    the firewall. If you are using \texttt{lemuria} you should use one of the port addresses
    that has been allocated to you for this course.}

\item In a separate window bring up the Chatter ``client'' program. You will need to provide it
  with the name of the host where the name server is running. For example
\begin{verbatim}
client vortex.cis.vtc.edu
\end{verbatim}

  Once the client is running you can use the \texttt{/name} command to set the nickname you want
  to use on your outgoing messages. Ordinarily this would be the same nickname that you are
  using with your server (which receives incoming messages) but the Chatter system does not
  require this\footnote{However, you \emph{must} use the same name in order for the chat rooms
    to work in a reasonable way!}. You can now use the \texttt{/msg} command to send messages to
  other Chatter uses. Use the \texttt{/help} command for a complete list of supported commands.

\item Use the \texttt{/rooms VTC} command to select the room manager named ``VTC.'' Use the
  \texttt{/create} or \texttt{/join} commands to create a new chat room or to join an existing
  chat room. You can use the \texttt{/list} command to get a list of the existing chat rooms.

\end{enumerate}

\section{Garble Filtering}

To illustrate the flexibility of this system you will next implement a Chatter filter class that
garbles text by applying a ROT13 encoding to it. You will set up an instance of the garble
filter to be an ordinary Chatter receiver that passes its output to a particular chat room
(named ``garble''). Ordinary users who join this chat room will be able to chat among themselves
normally, but any text sent to the garble filter will be broadcast to the entire room in ROT13
encoded form. While this application is rather silly, one could imagine the filter performing
some useful transformation on the incoming text.

\begin{enumerate}

\item Add the files \texttt{Garble.h} and \texttt{Garble.cpp} to the Chatter library folder. In
  \texttt{Garble.h} provide the declaration of class \texttt{Garble\_impl} that derives from
  \texttt{Chatter::TextFilter}. See \texttt{TextSink.h} for ideas about how to do this. In
  \texttt{Garble.cpp} implement the necessary methods to ROT13 encode each input line and pass
  the result on to the output. See the attached file \texttt{rot13.cpp} for a function that does
  ROT13 encoding. Modify the makefile(s) so that your new additions to the Chatter library are
  built when the rest of the system is built.

\item Create a new program in the \texttt{programs} folder named \texttt{garbler}. Your program
  should do the following.

\begin{enumerate}

\item Initialize Ice and create an object adapter. Create an instance of \texttt{Garble\_impl}
  and attach it to the adaptor. See \texttt{server.cpp} for specifics.

\item Look up the `garble' room on the VTC room manager (you will need to use the name server to
  get a proxy to the room manager) and connect the output of your garble filter to the garble
  room. See \texttt{client.cpp} for specifics.

\item Create a makefile so that your new program is built when the rest of the system is built.
  Copy the makefile from one of the existing programs to get started. Also modify the top level
  makefile appropriately.
\end{enumerate}

\end{enumerate}

When you are done with the steps above you should be able to issue a make operation at the top
level of the source tree and have it build your new garble filter class and garbler program.

\section{Demonstration}

Now demonstrate your system by bringing up your garbler program and having it register itself
with the name server under a special nickname (perhaps something like ``Peter-garble''). Run
your normal client and server programs and join the garble chat room. You should find that any
text sent to the garble filter appears in the chat room in ROT13 encoded form.

Note that this system illustrates the interaction of several programs working as peers. The
garbler program is particularly interesting because it is both a server and a client. The
various programs in the system communicate by invoking methods on each other's objects by way of
special pointers (called ``proxies'' in Ice terminology) that can effectively point at objects
on other machines. You should reflect on what would be involved in implementing this system
using traditional network programming techniques such as we discussed at the beginning of this
course.

\section{Report}

Write a formal report for this lab that describes how your extensions to the Chatter system
work. Include a listing of your changes to the source files.

\end{document}
