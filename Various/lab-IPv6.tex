%%%%%%%%%%%%%%%%%%%%%%%%%%%%%%%%%%%%%%%%%%%%%%%%%%%%%%%%%%%%%%%%%%%%%%%%%%%
% FILE         : lab05.tex
% AUTHOR       : (C) Copyright 2010 by Peter C. Chapin
% SUBJECT      : IPv6 lab for CIS-3152
%
% In this lab the students will experiment with IPv6 on their Windows stations.
%%%%%%%%%%%%%%%%%%%%%%%%%%%%%%%%%%%%%%%%%%%%%%%%%%%%%%%%%%%%%%%%%%%%%%%%%%%

% ++++++++++++++++++++++++++++++++
% Preamble and global declarations
% ++++++++++++++++++++++++++++++++
\documentclass[twocolumn]{article}

%\pagestyle{headings}
%\setlength{\parindent}{0em}
%\setlength{\parskip}{1.75ex plus0.5ex minus0.5ex}

% +++++++++++++++++++
% The document itself
% +++++++++++++++++++
\begin{document}

% ----------------------
% Title page information
% ----------------------
\title{CIS--3152 Lab \#5\\IP Version 6}
\author{Peter C. Chapin\thanks{Peter.Chapin@vtc.vsc.edu}\\
  Vermont Technical College}
\date{Last Revised: February 23, 2010}
\maketitle

\section{Introduction}

In this lab you will demonstrate IPv6 communication between Windows XP stations, and tunneled
IPv6 communications between Windows stations and a Unix server that are separated by an
IPv4-only network. You will observe these communications on the network analyzer.

\section{Setting up IPv6}

\emph{This section requires administrative access to the machine.} Since IPv6 is an independent
network protocol it must be installed along side the IPv4 protocol. In other words, having IPv4
installed does not imply that IPv6 is also installed. Starting with Windows XP, Microsoft
provides an IPv6 protocol driver ``out-of-the-box.'' However, IPv6 support is not turned on by
default in Windows XP. To activate it, perform the following steps:

\begin{enumerate}

\item In the Control Panel, open ``Network Connections.'' Right click on the ``Local Area
  Network'' connection and select ``Properties'' from the context menu.

\item Click the ``Install'' button and select ``Protocol''. Click the ``Add'' button to add a
  new protocol.

\item You should find ``Microsoft TCP/IP version 6'' as an option. Select it and click ``OK'' to
  install the protocol.

\end{enumerate}

After installing IPv6 it should show up in the list of installed protocols in the properties of
the local area connection. If you select it you will notice that you do not have the option of
clicking the ``Properties'' button. This implementation of IPv6 is not configurable in this way.
However, one advantage of the IPv6 protocol is that it can auto-configure. For example your
station will deduce its own link-local address and, using neighbor discovery protocol, will get
its site-local (if applicable) and global address prefixes from the nearest IPv6 router.

The Microsoft implementation of IPv6 is configurable, however, using certain command line tools.
At a command prompt, type \texttt{ipv6 if} to list the IPv6 interfaces on your machine. You
should find several of them. The interface labeled ``Local Area Connection'' is the real
interface. The other interfaces are all virtual interfaces used for special purposes. What is
the link-local address of your station on the LAN? You should also take note of the interface
index value of your LAN connection (the small integer used to number the interfaces).

You can use the \texttt{ping6} utility to ping another IPv6 station in the lab. This will verify
that basic IPv6 connectivity is available\footnote{Due to firewall configurations, \texttt{ping}
  might not respond even if IPv6 is working. However, you can use wireshark to observe the
  outgoing ping packets anyway.}.

\section{The Programs}

For this lab you can download from the class web site the source code for the IPv6 versions of
the daytime client and server. You should get both the client and server for both Windows and
Linux even though you may not ultimately need all four programs. You might want to experiment
with different combinations of client and server, however, especially to troubleshoot problems.
Review the code for these samples and compile them on their respective platforms.

\section{Procedure}

The procedure below asks you to record header information from several packets on the network.
Our interest in this lab is with the IP layer headers (not TCP nor Ethernet) so focus on those
headers only. You don't need to record every field, but certain source and destination addresses
should be noted. Some other fields might be interesting as well.

\begin{enumerate}

\item Your instructor will run an IPv6 version of the daytime server on the instructor's station
  in the lab. He will post the link-local address of that station on the board for reference.
  Demonstrate that the client works correctly with the server. Observe the connection with
  Wireshark and decode the IPv6 packets. Record the header information from a typical packet
  (for example the packet containing the client's initial SYN segment).

\item Compile the daytime server for Linux and run it on \texttt{lemuria}. Use the globally
  unique local address for
  \texttt{lemuria}\footnote{\texttt{FD25:F376:7B60:0001:021E:4FFF:FE30:D903}} and attempt to
  connect to it. Decode the network traffic using Ethereal. The connection should fail, but what
  do you see in the network analyzer?
  % Note that on the Windows machines in the lab, the SYN segment is sent to the default 6to4
  % gateway. I believe this is because the machines have no global address and thus don't know
  % what else to do with the packet. As a result the packet is encapsulated, which was not the
  % intent of this section.

\item Using the Linux server again, attempt to connect to it using the server's 6to4 address.
  Decode the network traffic and record the header information from a typical packet. Be sure to
  record information from both the IPv6 header and the enclosing IPv4 header.
  % Be sure the firewall on the Linux machine allows IPv6 connections to the appropriate port.
  % Note that settings in iptables are for IPv4. There is a separate ip6tables for IPv6.

\item Use your workstation's automatic tunneling interface to connect to the Linux server. Again
  observe the traffic and record the header information from a typical packet.
  % This section doesn't seem to work (at least I couldn't figure out how to make it work).

\item Use the IPv4 daytime client (from a previous lap) to connect to the Linux server. What
  address does the server display for the client? Once again observe the traffic and record the
  header information from a typical packet.
  % It's important that the students see an IPv6 server handling connections from IPv4 only
  % clients.

\end{enumerate}

\section{Report}

There is no formal report for this lab. Turn in the information you recorded in the procedure
above.

\appendix

\section{Configuring IPv6 Addresses (Linux)}

On a Linux system you can use \texttt{ifconfig} to associate an IPv6 address with a particular
interface. The syntax to use is as follows.
\begin{verbatim}
ifconfig eth0 inet6 <addr>/<prefix-length>
\end{verbatim}

\section{Configuring 6to4 Tunnels (Linux)}

This information describes how to configure a 6to4 tunnel endpoint on a Linux machine. It was
taken from the Linux IPv6 ``How To'' document.

\begin{enumerate}
\item Create a new tunnel device.
\begin{verbatim}
/sbin/ip tunnel add tun6to4 mode sit ttl \
  64 remote any local <localipv4addr>
\end{verbatim}

\item Bring the interface up.
\begin{verbatim}
/sbin/ip link set dev tun6to4 up
\end{verbatim}

\item Add local 6to4 address to the interface. Note that the prefix length \emph{must} be 16.
\begin{verbatim}
/sbin/ip -6 addr add <local6to4addr>/16 \
  dev tun6to4
\end{verbatim}

\item Add default route to the global IPv6 network using the all-6to4-routers IPv4 anycast
  address.
\begin{verbatim}
/sbin/ip -6 route add 2000::/3 via \
  ::192.88.99.1 dev tun6to4 metric 1
\end{verbatim}

  Note that some versions of \texttt{ip} (such as the version with SUSE Linux) does not support
  the IPv4-compatible IPv6 addresses for gateways. In that case the related IPv6 address of the
  gateway has to be used instead (\texttt{2002:C058:6301::1}).
\end{enumerate}

\end{document}

