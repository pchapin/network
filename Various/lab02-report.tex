%%%%%%%%%%%%%%%%%%%%%%%%%%%%%%%%%%%%%%%%%%%%%%%%%%%%%%%%%%%%%%%%%%%%%%%%%%%
% FILE    : lab02-report.tex
% SUBJECT : Sample report for the concurrent TCP server lab.
% AUTHOR  : (C) Copyright 2013 by Peter C. Chapin <PChapin@vtc.vsc.edu>
%
%%%%%%%%%%%%%%%%%%%%%%%%%%%%%%%%%%%%%%%%%%%%%%%%%%%%%%%%%%%%%%%%%%%%%%%%%%%

% ++++++++++++++++++++++++++++++++
% Preamble and global declarations
% ++++++++++++++++++++++++++++++++
\documentclass{article}

% --------
% Packages
% --------
\usepackage{fancyvrb}
\usepackage[pdftex]{graphicx}
\usepackage{listings}
\usepackage{hyperref}
\usepackage{url}

% The following are settings for the listings package
\lstset{language=C,
        basicstyle=\small,
        stringstyle=\ttfamily,
        commentstyle=\ttfamily,
        xleftmargin=0.25in,
        showstringspaces=false}


% ------------------
% Layout adjustments
% ------------------
% Avoid changing the layout; it is usually set via the document class (or in packages).
%
%\pagestyle{headings}
%\setlength{\parindent}{0em}
%\setlength{\parskip}{1.75ex plus0.5ex minus0.5ex}


% ------
% Macros
% ------
\newcommand{\filename}[1]{\texttt{#1}}
\newcommand{\command}[1]{\texttt{#1}}
\newcommand{\code}[1]{\texttt{#1}}


% +++++++++++++++++++
% The document itself
% +++++++++++++++++++
\begin{document}

% ----------------------
% Title page information
% ----------------------
\title{Lab \#2: Concurrent TCP Servers\\(Report)}
\author{Peter C. Chapin\thanks{PChapin@vtc.vsc.edu}\\
  Vermont Technical College}
\date{January 31, 2013}
\maketitle

% --------
% Abstract
% --------
\begin{abstract}
  I implemented a ``quote of the day'' server following the specification in RFC-865. The server
  supported concurrent client connections by creating a child process for each connection. I was
  able to clean up zombie children by using a handler for the SIGCHLD signal. The system behaved
  as expected.
\end{abstract}

% ------------
% Main Content
% ------------

\section{Introduction}
\label{sec:introduction}

The ``quote of the day'' protocol is described in RFC-865. That RFC describes both a connection
oriented (TCP) service and a datagram (UDP) service. In this lab I only implemented the TCP
service.

The protocol requires the server to send the quote to the client as soon as the client connects.
The client does not need to send any kind of request nor should it attempt to do so. Once the
quote has been sent, the server closes the connection. The client simply reads the connection
until it receives an end-of-file indication. It then closes the connection as well, completing
the transaction. In this respect the ``quote of the day'' protocol is very similar to the
daytime protocol.

Quotes should only contain ASCII printable characters. Lines should be terminated with a CR/LF
pair (my implementation insures this regardless of the server operating system). Quotes should
have a total length less than 512 bytes (my implementation does not check the length of the
quotes, but it could easily be modified to do so).

Notice that the ``quote of the day'' protocol would be appropriate for implementation by an
iterative server. This is because the server can execute the service quickly and never needs to
wait for the client. However as an exercise a concurrent server was implemented regardless.

\section{Server}
\label{sec:server}

The attached file, \filename{qotdd.cpp}, is my server program in C++. This program is a fairly
standard Unix concurrent server such as discussed in class. After some initialization tasks, it
listens to the specified port (17 by default) for incoming TCP connections. For each connection
that arrives, the server forks a child copy of itself to handle that connection.

Listing~\ref{lst:main-loop} shows the structure of the main loop of the program. Function
\code{accept} is called in a loop so that spurious returns with the \code{EINTR} error code are
ignored. This occurs when the signal handling function, described later, interrupts an execution
of \code{accept}.

\begin{figure}[tbhp]
\begin{lstlisting}[
  frame=single,
  xleftmargin=0in,
  caption={Main Loop of Server},
  label=lst:main-loop]
while (1) {
  memset(&client_address, 0, sizeof(client_address));
  client_length = sizeof(client_address);

  while ((connection_handle =
              accept(listen_handle,
                     (struct sockaddr *) &client_address,
                     &client_length)) < 0) {
    if (errno != EINTR) {
      perror("Problem accepting a connection");
      return 1;
    }
  }

  // Generate a random number...

  if ((child_ID = fork()) == -1) {
    perror("Problem creating child server");
    return 1;
  }
  else if (child_ID == 0) {
    close(listen_handle);
    // Send the quote of the day...
    close(connection_handle);
    exit(0);
  }
  close(connection_handle);
}
\end{lstlisting}    
\end{figure}

Once \code{accept} returns successfully the server uses \code{fork} to create a child process to
handle the connection. The child process inherits all the open socket handles of the parent and
thus can begin communicating with the client right away. It is important, however, that the
parent close the accepted \code{connection\_handle} so that when the child closes it, the
connection will terminate. I observed that if this was not done, the client would hang waiting
for possible data from the parent.

The parent server also processes the SIGCHLD signal by waiting for all zombie children and thus
eliminating them. Listing~\ref{lst:signal-handler} shows the signal handler I used. Notice that
the handler loops so that it can collect all zombies that have accumulated between the time
SIGCHLD is first delivered and the signal handler executes. This is necessary because signals
are not queued. Only one SIGCHLD will be delivered to the server process even if many children
terminate before the first one can be cleaned up.

\begin{figure}[tbhp]
\begin{lstlisting}[
  frame=single,
  xleftmargin=0in,
  caption={Signal Handler},
  label=lst:signal-handler]
void SIGCHLD_handler(int)
{
  int status;

  while (waitpid(-1, &status, WNOHANG) > 0) ;
  return;
}
\end{lstlisting}    
\end{figure}

Listing~\ref{lst:initialize} shows the code I used in the \code{main} function to initialize
signal handling. I also included the header \code{<signal.h>} to declare the various signal
handling functions.

\begin{figure}[tbhp]
\begin{lstlisting}[
  frame=single,
  xleftmargin=0in,
  caption={Signal Handling Initialization},
  label=lst:initialize]
action.sa_handler = SIGCHLD_handler;
sigemptyset(&action.sa_mask);
action.sa_flags = 0;
sigaction(SIGCHLD, &action, &old_action);
\end{lstlisting}    
\end{figure}

To make it easier to install new quotes, at startup the server reads a text file that contains
the quotes it will use. This is done in function \code{read\_quotes}. The server stores these
quotes in a \code{std::vector<std::string>} where each vector element contains a single quote.
If a quote has more than one line, then CR/LF pairs are embedded in the string as appropriate.
When a client connects, my server selects one of the quotes at random to send.

I noticed that it was important for the server to generate a random number for each client in
the parent process and not in the child processes. If the random number was generated in the
child processes, then each client got the same quote. This occurred because the state of the
random number generator in the parent was never modified. When each child forked it got a fresh
copy of the same state and hence generated the same number. By generating all the random numbers
in the parent before the fork, the state of the random number generator was able to evolve
normally and each child used a different number as desired.

A more deluxe version of the server should probably also support the ability to reread the
quotes file upon receiving a SIGHUP signal. In addition, the name of the quotes file should be a
command line argument and not built into the software.

\section{Testing}
\label{sec:testing}

To test the concurrent behavior of the server I first made a temporary modification whereby each
child process printed its process ID number and also sent that ID number to the client. In
addition I added an artificial 30 second delay so that I could more easily start multiple
``simultaneous'' clients. Listing~\ref{lst:modifications} shows both the overall child process
with the modifications and the original code that sends a quote. Notice that the modified code
is only compiled when the symbol \code{DEBUG} is active.

\begin{figure}[tbhp]
\begin{lstlisting}[
  frame=single,
  xleftmargin=0in,
  caption={Server Child Modifications},
  label=lst:modifications]
close(listen_handle);

#ifdef DEBUG
printf("\tServicing client with process: %d\n", getpid());
sleep(30);
snprintf(buffer, BUFFER_SIZE, "From process %d: ", getpid());
write(connection_handle, buffer, strlen(buffer));
#endif

// Send the quote of the day.
write(connection_handle,
      quotes[random].c_str(), quotes[random].size());
close(connection_handle);
exit(0);
\end{lstlisting}    
\end{figure}

Figure~\ref{fig:sample-run} shows a sample run of the server and two clients that connected at
the same time. The bottom of the figure shows the abbreviated output of \command{ps aux} between
when the clients first connected and when they were serviced. This output shows the main server
process with PID 3468 and the two child processes. After the clients terminated a similar
listing showed only the main server process; the child process zombies were not present.

\begin{figure}[tbhp]
\begin{Verbatim}[fontsize=\small, frame=single, commandchars=\\\{\}]
peter@whirlwind:~/qotd/server/bin/Debug$ ./server 9000
Read 385 quotes.
Accepted client connection from: 127.0.0.1
	Servicing client with process: 3473
Accepted client connection from: 127.0.0.1
	Servicing client with process: 3475


peter@whirlwind:~/qotd/client/bin/Debug$ ./client 127.0.0.2 9000
From process 3473: What's the difference between roast beef and
pea soup? Anyone can roast beef.


peter@whirlwind:~/qotd/client/bin/Debug$ ./client 127.0.0.2 9000
From process 3475: Thank God for the IRS  -  Without them I'd be
stinking rich!


peter     3468  ./server 9000
peter     3473  ./server 9000
peter     3475  ./server 9000
\end{Verbatim}
\caption{Sample Run}
\label{fig:sample-run}
\end{figure}

To illustrate the effect of the signal handler I temporarily commented out the signal handling
initialization code and reran the experiment above. The behavior was similar except that even
after the clients terminated the output of \command{ps aux} showed zombie children.

\section{Conclusion}
\label{sec:conclusion}

My server program worked as desired. It supported multiple clients simultaneously and it
correctly removed the zombies arising from terminated child servers.

\end{document}
