%%%%%%%%%%%%%%%%%%%%%%%%%%%%%%%%%%%%%%%%%%%%%%%%%%%%%%%%%%%%%%%%%%%%%%%%%%%%
% FILE         : exam-02b.tex
% AUTHOR       : Peter C. Chapin
% LAST REVISED : 2013-04-25
% SUBJECT      : Exam #2 for CIS-3152
%
% Send comments or bug reports to:
%
%       Peter C. Chapin
%       Computer Information Systems Department
%       Vermont Technical College
%       Williston, VT 05495
%       PChapin@vtc.vsc.edu
%%%%%%%%%%%%%%%%%%%%%%%%%%%%%%%%%%%%%%%%%%%%%%%%%%%%%%%%%%%%%%%%%%%%%%%%%%%%

\documentclass[12pt]{examdesign}

% Options.
\Fullpages
\NoRearrange
% \NoKey
\NumberOfVersions{1}

% Class and exam information.
\class{CIS-3152}
\examname{Exam \#2}

% This setting seems to be ignored (in the answer key at least).
% \setlength{\parskip}{1.75ex plus0.5ex minus0.5ex}

\begin{document}

% I can't seem to be able to use namedata@vspace below (see documen-
% tation). I can use the default \namedata but I can't seem to redefine
% \namedata to suite my needs.
%
\begin{examtop}
  \parbox{3in}{\classdata \\
               \examtype, Form: \fbox{\textsf{\Alph{version}}}}
  \hfill
  \parbox{3in}{\normalsize Name: \hrulefill \\[2.0ex]
                           Date: \hrulefill }
  \bigskip
\end{examtop}

\textit{This is a take-home exam. Do not work with anyone else on the exam or post questions
  about the exam to any Internet forums, etc. You can use any other resources including the
  text, your notes, or documents you find online (including old forum discussions started by
  others). Please include a URL for any Internet resources you use. If you have any questions
  about the exam, feel free to contact me.}

\begin{shortanswer}

  % SMTP
  \begin{question}
    In the SMTP protocol the client and server exchange lines of US-ASCII text. One could
    imagine a binary mail transport protocol as an alternative: single byte control codes could
    be used to mark significant information (from address, to address, etc) and the mail
    messages themselves could be allowed to contain arbitrary binary data. What are the
    advantages and disadvantages of such an approach?

    \begin{answer}
      The main advantage of using a binary mail protocol is that the data transmitted on the
      wire would be more compact and thus could be transmitted more efficiently. With less
      overhead there would be higher effective bandwidth.
      
      The main disadvantage, aside from incompatibility with current mail servers, is that it
      would no longer be possible to just telnet to a server and interact with it directly. In
      other words, testing and debugging mail servers would be more difficult.
    \end{answer}
  \end{question}

%\pagebreak

  % RFC-5322
  \begin{question}
    In this class we presented the structure of email messages (as described in RFC-5322) as a
    kind of protocol layer above SMTP and below MIME. Explain what that means.

    \begin{answer}
      The lower level protocol, SMTP, does not need to know about the details of how mail
      messages are formatted. Similarly mail messages do not need to take into account the MIME
      structuring that occurs at the upper level. In other words SMTP, RFC-2822, and MIME are
      independent of each other. Each protocol ``nests'' inside the lower one. This is exactly
      how, for example, IP and TCP are related so in that sense the three protocols are layered
      in a similar way.
    \end{answer}
  \end{question}

%\pagebreak

  % MIME
  \begin{question}
    Consider the attached multipart/mixed email message. Answer the following questions.
    \begin{enumerate}
      \item How many parts does the message have and what is the MIME type of each part?
      \item The encoding of the second attachment ends with an `=' character. What is the
      significance of that? Why doesn't the encoding of the first attachment end with an `='
      character?
      \item Suppose you modified the second occurence of the boundary line (the one just before
      the first attachment) so that it no longer matched the other boundary lines? What would
      happen to the message structure and how would you expect the message to be displayed?
    \end{enumerate}
    
    \begin{answer}
      \begin{enumerate}
      
        \item There are three parts. The first is text/plain, the second is text/plain, and the
        third is image/jpeg.
        \item The trailing `=' is a padding character used as part of the base64 encoding. Such
        characters are needed when the input file is not a multiple of three in length. The
        first attachment didn't need any such padding because, apparently, it is a multiple of
        three in length.
        \item Instead of having two attachments there would be only one (image/jpeg) attachment.
        The encoded text of the original first attachment (including the modified boundary line)
        would then be interpreted as text in the first part of the message.
      \end{enumerate}
    \end{answer}
  \end{question}

%\pagebreak

  % XML Schema
  \begin{question}
    Consider the attached sample XML document. This document is an instance of a hypothetical
    exam markup language that can be used to structure tests in courses such as this one. Using
    the W3C's XML Schema language write the definition for the \texttt{question} element. You
    can assume that the \texttt{xs} name space prefix has already been declared (as was done in
    lab). Notice that a question can have multiple answers. You do not need to write an entire
    schema.
  \end{question}

%\pagebreak

%  % XPath/XSL
%  \begin{question}
%	  Again consider the attached sample XML document. Answer the following questions.
%	  \begin{enumerate}
%	    \item Show an absolute XPath expression that selects all questions with a difficulty
%	    of 3.
%	    \item Assuming that the context node is a particular question, show a relative XPath
%	    expression that selects the second answer (if any) of that question.
%	    \item Write an XSLT template that converts a question element into XHTML. Wrap the
%	    body of the question and the answer inside XHTML paragraphs ($<$p$>$). You do not
%	    need to process the attributes of the question element.
%	  \end{enumerate}
%  \end{question}

  % RPC
  \begin{question}
    All remote procedure call systems attempt to ``hide the network.'' Explain what this means.
    Give at least one way in which this hiding is imperfect.
  \end{question}

\end{shortanswer}
\end{document}
