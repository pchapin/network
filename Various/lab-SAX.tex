%%%%%%%%%%%%%%%%%%%%%%%%%%%%%%%%%%%%%%%%%%%%%%%%%%%%%%%%%%%%%%%%%%%%%%%%%%%
% FILE         : lab08.tex
% AUTHOR       : (C) Copyright 2007 by Peter C. Chapin
% LAST REVISED : 2007-03-17
% SUBJECT      : SAX
%
% In this lab the students will build a small application using SAX to
% process an XML document.
%
%       Peter C. Chapin
%       Electrical and Computer Engineering Technology
%       Vermont Technical College
%       Randolph Center, VT. 05061
%       pchapin@ecet.vtc.edu
%%%%%%%%%%%%%%%%%%%%%%%%%%%%%%%%%%%%%%%%%%%%%%%%%%%%%%%%%%%%%%%%%%%%%%%%%%%

% ++++++++++++++++++++++++++++++++
% Preamble and global declarations
% ++++++++++++++++++++++++++++++++
\documentclass[twocolumn]{article}
% \pagestyle{headings}
\setlength{\parindent}{0em}
\setlength{\parskip}{1.75ex plus0.5ex minus0.5ex}

% +++++++++++++++++++
% The document itself
% +++++++++++++++++++
\begin{document}

% ----------------------
% Title page information
% ----------------------
\title{CIS--3152 Lab \#8\\SAX}
\author{\copyright\ Copyright 2007 by Peter C. Chapin}
\date{Last Revised: March 17, 2007}
\maketitle

\section*{Introduction}

In this lab you will write a short program that uses an off-the-shelf XML parser to process an
XML document. In many cases one can write an XSLT style sheet that processes an XML document to
produce results that are also in XML form. Normally writing an XSLT style sheet is easier than
writing a specialized program from scratch. However, there are times when a specialized program
is necessary. For example, if the output you desire is not in XML (tree-like) form, XSLT isn't
such a good choice. Also general XSLT processors might be too slow in some cases. In this lab
you will write a custom program to process XEmail documents.

\section{The Problem}

Consider the XEmail markup language that you worked on in the last lab. Write a program that
produces a summary listing of all unread email messages in a document. The summary listing
should include the From address on the message and the message's subject line. Your program
should also print a count of the number of such messages. Note that messages that have been read
have a status attribute \emph{that contains} ``R.''

\section{Implementation Notes}

In the XEmail project you will find a sample SAX parsing application written in C++ that uses
the Xerces parser. The sample application contains all the necessary boiler plate to do a
validating parse on an XML file. The sample also contains place holder functions for handling
significant SAX events. The provided Makefile builds the sample. Currently the sample just
outputs a listing of the elements encountered during parsing.

To modify the sample application so that it meets the needs of this lab, do the following:
\begin{enumerate}

\item Change \texttt{startElement} so that it sets a flag when it sees an \texttt{email:message}
  element without the R status. Change \texttt{endElement} so that it clears this flag when the
  end of the \texttt{email:message} element is encountered.

\item You need to detect `From' addresses and `Subject' lines only when the flag mentioned above
  is set (that is, only when you are inside an appropriate message). Because there are multiple
  places where \texttt{address} elements might be encountered, you may need to use additional
  flags to distinguish `From' addresses from, say, `To' addresses. A more elegant way is to
  maintain a stack of the elements that are currently open and then do a kind of string matching
  against the top of the stack. Such an approach might be overkill for this short problem,
  however.
\end{enumerate}

\section{Report}

Turn in a commented listing of your program. There is no formal report.

\end{document}
