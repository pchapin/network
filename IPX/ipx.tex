%%%%%%%%%%%%%%%%%%%%%%%%%%%%%%%%%%%%%%%%%%%%%%%%%%%%%%%%%%%%%%%%%%%%%%%%%%%
% FILE         : ipx.tex
% AUTHOR       : Peter Chapin
% LAST REVISED : April 28, 2000
% SUBJECT      : IPX programming problem for EL-330
%
%
% Send comments to Peter Chapin (pchapin@night.vtc.vsc.edu) or at the
%   snail mail address of:
%
%       Peter Chapin
%       Vermont Technical College
%       Main Street
%       Randolph Center, VT. 05061
%
%%%%%%%%%%%%%%%%%%%%%%%%%%%%%%%%%%%%%%%%%%%%%%%%%%%%%%%%%%%%%%%%%%%%%%%%%%%

% ++++++++++++++++++++++++++++++++
% Preamble and global declarations
% ++++++++++++++++++++++++++++++++
\documentclass{article}
% \pagestyle{headings}
\setlength{\parindent}{0em}
\setlength{\parskip}{1.75ex plus0.5ex minus0.5ex}

% +++++++++++++++++++
% The document itself
% +++++++++++++++++++
\begin{document}

% ----------------------
% Title page information
% ----------------------
\title{IPX Programming Problem}
\author{Peter Chapin}
\date{Last Revised: April 28, 2000}
\maketitle

\section{Introduction}

This document describes a programming problem that will give you some
direct exposure to how the IPX protocol is handled by the software on a
network station. In this exercise, you will use the Novell NetWare
Software Development Kit (SDK). This package contains a large number of
functions for interacting with NetWare networks. Since NetWare uses IPX,
Novell supports the protocol fully in their products.

\section{Preliminaries}

The NetWare SDK provides libraries suitable for linkage to programs
compiled with either Borland C++ or Microsoft's Visual C++. It does not
appear to be an easy matter to uses these libraries with programs
compiled with Watcom C++. It may be possible to arrange it, but doing so
would be more work than we want to do at this time.

To use Borland C++ v4.52 on our network, type the following.

\begin{verbatim}
use bcpp
\end{verbatim}

The command line compiler is ``bcc.''

The NetWare SDK itself is located in s:$\backslash$ec$\backslash$nwsdk.
The documentation is in Windows help files. Consult the help for 16 bit
clients (we will be creating an MS-DOS program).

When you compile a program that uses the SDK, you will need to direct
the compiler to scan the SDK's include directory. This is where the
various header files are located. That directory is
s:$\backslash$ec$\backslash$nwsdk$\backslash$include. Use the ``-I''
command line option on the compiler to specify this directory.

When you link your program, you must also tell the linker to scan over
the appropriate LIB file that contains the executable code for the
functions you want to use. This file is
s:$\backslash$ec$\backslash$nwsdk$\backslash$lib$\backslash$dos$\backslash$borland$\backslash$SNWIPXSP.LIB.
Mention this file on the command line to cause the linker to scan over
it.

Finally you also need to \#define the symbol {\tt NWDOS} so that the
proper declarations in the SDK header files will be seen by the
compiler. The NetWare SDK supports programming under DOS, Windows,
Win95, WinNT, and OS/2. It is necessary to specify the platform so that
platform specific matters can be properly handled in the header files.
Putting all this together leads to a rather long command line.

\begin{verbatim}
bcc -DNWDOS -Is:\ec\nwsdk\include myprog.c \
  s:\ec\nwsdk\lib\dos\borland\snwipxsp.lib
\end{verbatim}

You may want to create a batch file to simplify your life.

\section{The Programs}

Attached you will find the programs sendfile.c and recvfile.c. Modify
recvfile so that it properly terminates when it received the zero code.
Compile the programs and try them out. You will have to change the
network number to 22 (for G-111) and the node number to an appropriate
value for one of the stations in that lab. Be sure to only send text
files using these programs.

You should notice that when sending large files, some data is lost. This
is because there is no flow control in these programs. To deal with this
issue, modify the programs so that the receiver sends an acknowledgement
packet back to the sender for each packet it receives. The sender should
not send another packet until it gets the acknowledgment.

You can copy the {\tt Recv} function from the receiver into the sender
program. You will also have to copy significant parts of the sender into
the receiver program as well so that the receiver can send an
acknowledgment.

\end{document}
